\section{大西洋革命}

\subsection{导论}

\subsection{美国独立革命}

\subsubsection{背景}
17世纪初,英国人开始在北美大西洋沿岸建立殖民地。100多年以后,英国在北美的殖民地已经有13个。这些殖民地居民除英国移民和土著居民印第安人外,还有来自欧洲其他国家的人以及非洲来的黑人奴隶。18世纪中期,北美经济有了一定的发展,美利坚民族开始形成。

殖民地政权来自英王特许,并且各殖民地分别独立。殖民地比较民主,不存在封建特权和等级制度。下层的劳动群众为新教。

\subsubsection{殖民地的发展}
英属北美殖民地主要分为三个部分,分别是北部新英格兰地区,中部与南部。北部资本主义工商业比较发达,造船业是主要的工业部门之一。中部殖民地盛产粮食,生产的小麦和玉米远销欧洲市场,被誉为“面包殖民地”。南部殖民地种植园盛行,黑人奴隶是种植园的主要劳动力,除生产稻米外,主要种植烟草,棉花等经济作物,实行种植园奴隶制。

1756~1763的“七年战争”,英国从法国夺得土地。英国开始对北美殖民地征税,包括食糖法和印花税法。但根据英国法律在议院没有代表权不得纳税,于是集合反英集会。1765年10月召开反印花税会,征税问题由经济问题转移到政治层面,国王迫于压力取消了税款。

后由于东印度公司濒临破产,英国允许东印度公司向殖民地倾轧茶叶。殖民地人民进行了反抗,于是在1773年发生了著名的波士顿倾茶事件,激怒了英王,连续颁布了五项高压法令《波士顿港口法案》《马萨诸塞政府法》《司法管理法》《军队驻扎法》《魁北克法》,严重损害了殖民地人民利益,成为了导火索。

殖民地召开了第一届大陆会议,向国王提出情愿,政治上仍未宣布独立,但仍促使矛盾激化。
1775年4月“列克星敦枪声”成为导火索,英王企图镇压殖民地人民的武装,打响了第一枪。从此,北美殖民地人民争取独立的战争开始了。不久,各殖民地代表举行第二次大陆会议,决定组建军队同英国军队战斗,乔治·华盛顿被任命为总司令进行指挥,尽力将松散的民兵队伍整成严谨军纪的队伍,但此时殖民地仍未宣扬独立。对于1763年英王的错误主要在于两个,一个是征税,另一个是不明白北美殖民地人民需要什么,反抗什么,根本不允许殖民地平等地位,这个分歧直接导致了英国与殖民地之间的战争。

之后托马斯·潘恩发表《常识》,使舆论向赞成独立方向发展,极大地鼓舞了人们对独立的热情,开始起草独立宣言。

1776年7月4日,大会会议通过《独立宣言》,由托马斯·杰斐逊,起草《独立宣言》,之后正式开始独立战争。并在独立后获得了其它国家的支持,如法国的支持。7月4日,美国独立日。
开始时美军屡屡失利,但华盛顿鼓舞士气,加紧训练。1777年萨拉托加战役中美国打败英军,是美国独立战争的转折点。此后,法国等国家开始对美国提供军事援助。持续6年至1781年10月19日,英军将领康华里投降,1783年9月3日签订《巴黎合约》,英国正式承认美利坚合众国独立。

美国独立具有广泛的国际影响,结束了殖民统治,实现了国家独立,确立了比较民主的资产阶级政治体制,有利于美国资本主义的发展,对以后欧洲和拉丁美洲的革命也起了推动作用。直接影响了法国大革命的爆发和英国的议会改革。

\subsubsection{美国联邦制的形成}
美国最初时为邦联条例,《邦联条例》制定于1777年,内容说明邦联国会具有宣战,缔约,举债,招募军队等权力,但无征税权,无最高元首。各州保留了很大的独立性,有征税,征兵和发行纸币的权力。

邦联制的美国是一个由13州组成的松散联盟。主要反映了民主派种植园主的观点。

邦联制面临的问题有:1.邦联没有财政税收的权力,国家无力偿还战争留下的巨额内外债务;2。政府不能有效地促进国内外贸易;3.政府无力维持强大的常备军,也难以保障局势的稳定和国家的安全。

1787年5月25至9月15日,在费城召开制宪会议,抛弃了原有的邦联制度(与会人员均为有产者,有着基本一致的经济利益,这也是会议达成共识的关键),解决了三大问题:1.中央与州权关系问题;2.中央三大部门的关系问题;3.北部工商业阶级和南方奴隶主利益问题。最终签订了美国1787年宪法。

美国1787年宪法是世界上第一部成文宪法,确立了美国的联邦制,三权分立制,开创了世界的先例。

\subsubsection{美国1787年宪法}
主要内容:1.确立了美国的联邦体制;2.“三权分立”原则。内容包括

1.最高立法机构国会由参议院和众议院组成,参议院每州选举2人,任期6年,每两年改选总数的三分之一;众议员根据各州的人口比例选出,任期2年。

2.最高行政权属于总统,兼武装部总司令。总统由选举产生,任期4年,有权缔约,有权任命大使,最高法院法官和政府其他官员的权利,但需国会同意;总统对宪法负责,不对国会负责,对国会的立法有否决权,但国会在复议后仍以三分之二以上人数再次通过此法,即有效;

3.司法权属于各级法院,最高法官由总统任命,终身任职,具有解释宪法的权力。

宪法的不足在于承认了奴隶制的合法性。

华盛顿就职美国第一届总统(89年4月30日),担任了两届美国总统(1789~1797)后, 坚决不参加第三次总统竞选,退休后回到自己农庄。

\subsection{法国大革命}
背景是18世纪末落的封建专制制度,导火索是三级会议的召开(由于几代国王面临财政危机,加之法国支援美国而债台高筑,国王急于加税)
第一等级:天主教士;第二等级:贵族;第三等级:平民(资产阶级)

1789年5月5日,三级会议在凡尔赛宫召开。在启蒙思想熏陶下,第三等级广泛接受进步思想。在会议中要求制定宪法,限制王权。后成立国民议会,后来成为国民制宪会议。国王用无力镇压,后发生巴黎起义,攻占巴士底狱,标志着法国资产阶级革命的爆发(1789年7月14日),胜利后各地纷纷效仿,揭竿而起。

君主立宪派的统治(1789年7月~1792年8月)时期颁布以下法令:1.制宪会议成为最高权利机关;2.国民自卫军;3.拉法叶特——“两个世界的英雄”(建立波旁家族,举起象征革命的三色旗)。
“八月法令”废除了封建特权,颁布《人权与公民权宣言》,《人权宣言》,内容包括天赋人权,主权在民,法治原则和财产权原则。成为资产阶级革命的纲领性文件。“雅各宾俱乐部”讨论政策并指定法律。1791年1月20,国王逃跑。

《1791年宪法》内容包括三权分立,君主立宪制,行政权属于世袭的国王,立法权属于一院制的立法议会,司法权属于选举产生的法官。
具有一定的历史进步性,但具有较大的妥协性,保留了旧王朝的国王,确认财产的不平等,分“积极公民”和“消极公民”。确立了君主立宪政体。

1792年普奥联军攻入巴黎,人民奋起反抗,编写《马赛曲》,最后作为法国国歌。因为国王的背叛,人民掀起第二次法国革命,逮捕了国王,结束了君主立宪政体。建立了巴黎会议,打退了普奥联军入侵后,由吉伦特党人执政(1792年8月至1793年6月)

1792年9月21日,国民公会开幕,宣布废除君主政体;22日宣布成立共和国,即法兰西第一共和国;吉伦特派与雅各宾派的斗争亦趋激烈。路易十六被推上断头台,之后吉伦特派下台,雅各宾派上台。指定1793年宪法,是当时最民主的宪法,但迫于国内国外形势,采取了恐怖统治(马拉之死)。

罗伯斯庇尔等人采取的政策包括限定生活必需品的价格,管制粮食买卖,把没收来的逃亡贵族的土地以分期付款的方式卖给缺地的农民,同时,用恐怖手段严惩发动分子,但也伤及无辜。建立有巴黎“无套裤汉”组成的“革命军”,专门镇压与革命为敌者和投机商;《严惩嫌疑犯法令》;实行全面限价政策,规定对40种生活必需品实行最高限价;建立革命政府的决议。实质:暂时牺牲资产阶级利益,以换取民众支持。

\subsubsection{对雅各宾恐怖统治的分析}
实质:一种资产姐姐革命中出现的“非资产阶级方式”,即“平民方式”,也就是一种特殊的战时体制的资产阶级专政措施。

效果是经济状况好转,平定了国内的叛乱,击退了外国的入侵者。过激行为是杀人太多,破坏了法制;“非基督教化”运动,创立了共和国历法;(12个月的名称:葡月,雾月,霜月(秋季);雪月,雨月,风月(冬季);芽月,花月,牧月(春季);荻月,热月,果月(夏季))。
颁布牧月法令后,进一步扩大了恐怖。

“热月政变”——罗伯斯庇尔断头,雅各宾派终结。

\subsubsection{热月党人和督政府的统治(1795-1799)}
热月党的客观任务,是巩固资产阶级的既得成果,建立资产姐姐正常统治秩序,从政治和经济两方面结束恐怖。政治上取消恐怖政策,经济上废除限价政策,恢复经济自由。宗教上在坚持不发还教会地产和教会必须服从政府法令的条件下,允许天主教重新活动。

共和三年宪法(1795年宪法)国民议会解散。

\subsubsection{督政府的统治}
重建了从中央到地方的行政体系,进行币制改革和财政改革,统一了税收制度,鼓励发展经济。

对外战争,击败了第一次反法联盟,拿破仑军工显赫。拿破仑在意大利北部的阿尔克勒战役获胜。之后法国建立了军事独裁统治,来对抗联盟。“雾月政变”,1799年11月,拿破仑执政,开始了拿破仑帝国统治时期。

\subsection{拿破仑帝国的建立}
1800年“共和八年宪法”(在这之后打败反法第一次联盟),拿破仑·波拿巴(1769-1821)一步一步登上帝位的政策。

1802年“共和国十年宪法”;1804年11月,世袭制的“共和十二年宪法”通过,法国成为帝国。12月2日,拿破仑在巴黎圣母院举行加冕典礼,史称“拿破仑一世”。至此,法兰西第一共和国被法兰西第一帝国取代。
拿破仑治国方略:1.建立高效集权的国家机器;2.重视军队建设;3.平息叛乱,反对封建复辟;4.发展经济;5.实行开明的宗教政策;6.进行教育改革。

建立和完善了资产阶级法律体系。

颁布了五部法典:《法国民法典》(1804),《民事诉讼法典》(1806),《商业法典》(1807),《刑事诉讼法典》(1808),《刑法典》(1810)。其中《法国民法典》最为重要,拿破仑亲自参与了其中大部分修订。法典用资产阶级法制代替愚昧,为今后欧洲的法律体系提供了范本。拿破仑的一系列措施促进了资本主义工商业的发展,为法国资本主义发展奠定了基础。

\subsection{拿破仑战争}
拿破仑的总体目标是打败欧洲反法力量,确立法国在欧洲的霸主地位。同时,打击干涉军,维护革命成果和法兰西的独立。1807年后逐渐转为扩张和争霸。

由于反法联盟的不断干扰,拿破仑一次又一次地打退联盟军,并迫使联盟军签订条约。英国作为资产阶级兴起于国家,却与封建势力结盟,企图摧毁法国的资产阶级萌芽。

拿破仑实行了大陆封锁政策。制定《大陆封锁令》。英国对法国及其盟国实行反封锁,使法国的对外贸易和工业生产受到很大影响。造成拿破仑帝国发展的重重矛盾。拿破仑的大陆封锁政策作用微乎其微,主要是在英国的大陆封锁中,由于英国经济势力雄厚,并在海外拥有多个殖民地,从而大陆封锁政策作用微乎其微。

拿破仑战争的双重性在于拿破仑战争捍卫了法兰西民族的独立,巩固了资产阶级革命的社会成果。在被占领的欧洲地区,拿破仑进行了资产阶级性质的改革。拿破仑战争有力地破坏了欧洲的封建秩序,促进了资本主义的发展。同时带有侵略,扩张和争霸的性质。

\subsection{拿破仑帝国的危机与崩溃}
拿破仑的扩张和侵略所造就的帝国建立在强权政治和无力征服上,因为侵略地区人民的防抗不变。并且在扩张过程中与封建势力妥协,使得帝国的政治基础发生动摇。

由盛转衰的关键(1812年):远征俄国失利,元气大伤。反法势力趁机组成第六次反法联盟,攻入巴黎。1814年3月31日,拿破仑签署退位诏书,被流放到厄尔巴岛。反法势力扶持路易十八上台,在国内封建势力影响下,法国封建王朝复辟,路易十八实行发动统治,人民怨声载道。在该背景下,反法势力召开维也纳会议,企图就战后结果做妥协。拿破仑趁机从厄尔巴岛潜还至巴黎。

百日王(1815年3月20日~6月22日)

反法势力匆忙结成第七次联盟,在滑铁卢彻底终结了拿破仑传奇生涯,流放到圣勒拿岛,波旁王朝第二次复辟。

\subsection{拿破仑时代的欧洲}
欧洲革命和变革的推动力

在合并地区全面推行新制度(由于比较彻底地清除了封建制度,因为在这些地区资本主义发育良好,成为以后工业革命的领头军)。同时进行卫星国改造;反法联盟的改革(由于反法联盟的战败,英国,普鲁士等国不得不进行一些社会改革。所以客观上促进了欧洲社会的变革。)

\subsection{维也纳体系}
击败拿破仑之后召开维也纳会议(1814年10月1日~1815年6月9日),会议空前,领导的大国包括俄,普,奥,英,法。

奥地利首相美特涅;法国外交家塔列朗。会议中额“正统原则”+“补偿原则”,重绘欧洲政治地图。





