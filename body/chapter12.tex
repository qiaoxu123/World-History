\section{第一次世界大战及战后国家格局}

\subsection{第一次世界大战的爆发}
背景:第二次工业革命引起大国的经济政治发展不平衡,实力对比发生变化,新兴资本主义国家强烈要求按照实力重新瓜分世界。帝国主义国家之间的矛盾十分尖锐。后起的资本主义国家,如德国,要求重新分割世界,与英,法,俄等老牌资本主义国家展开了激烈的争夺。争夺霸权的结果,形成了两大敌对的帝国主义侵略集团——德国,奥匈帝国,意大利组成的三国联盟和英,法,俄组成的三国协约。这两大军事集团展开疯狂的扩军备战,世界大战一触即发。

德国在俾斯麦下台后,改变了原有的大陆政策,开始了新的“全世界”政策,即与其它国家争夺海外殖民地。

同时各国之间存在的矛盾与利益很快促成了国家之间的联盟,即德奥意大利为代表同盟国,英法俄为代表的协约国。

导火索:奥匈帝国皇储斐迪南大公夫妇在波斯尼亚首都萨拉热窝被暗杀。各国并没有采取缓和措施来化解矛盾,反而更激化了矛盾,产生了局部战争。更加深了帝国主义的战斗情绪。1914年7月底,奥匈帝国向塞尔维亚宣战。第一次世界大战爆发。德国与奥匈帝国为了加强同俄国,法国的对抗,1879年缔结了军事同盟条约。1882年,意大利同法国的矛盾加剧,也加入了德奥军事同盟,德奥意三国同盟形成。三国同盟的核心是德国。20世纪初,英国与德国的矛盾激化,兴国协调了与法,俄的关系,在1904年和1907年分别与法国和俄国签订协约,组后形成了三国协约。

各国宣战:德国于8月1日和3日分别对俄,对法宣战;8月4日英国对德宣战;8月6日奥匈帝国对俄国宣战。由于英日同盟,日本也对德宣战。以欧洲为战场的第一次世界大战开始了。

\subsection{第一次世界大战的经过}
为四个战线:西线,东线,巴尔干战线,意大利战线

马恩河战役中德国战败,使德国速战的想法破产;

历经10个月之久的凡尔登战役后,同盟国丧失主动权;

原本中立的美国在德国开始潜艇时加入协约国,对德宣战,进行反潜艇作战,德奥集团处于劣势。

俄国十月革命后签订《布列斯特和约》,退出战争;

签订贡比涅协定。

第一次世界大战规模空前。战场由最初的欧洲扩大到非洲,亚洲和太平洋地区,前后有30多个国家卷入战争。1917年,美国对德宣战,参加协约国;中国也加入协约国方面作战。大战末期,同盟国方面的国家逐渐支撑不住,先后投降。1918年11月,德国投降,历时4年多的第一次世界大战以同盟国的失败告终。结束了欧洲的全权霸主地位,经济惨遭破坏。建立了《凡尔赛·华盛顿》体系。

\subsection{俄国十月革命}
背景:沙俄帝国矛盾丛生,危机四伏。加上第一次世界大战对俄国的影响。社会矛盾引起工人起义,士兵起义各方面结合,最终爆发二月革命,结束了沙皇体制(二月革命(俄历2月23~3月2日)),沙皇退位,俄国建立了资产阶级临时政府,与二月革命中建立的工人士兵代表苏维埃同时存在,临时政府掌握着主要权力。

“苏维埃”是俄语cobet的音译,是“会议”或“代表会议”的意思。俄国二月革命中,工人和士兵建立了新的革命政权——工人士兵代表苏维埃。二月革命后,俄国建立了资产阶级临时政府,形成了两个政权并存的局面。临时政府企图消灭苏维埃建立自己的“一统天下”。列宁既是提出了“一切权利归苏维埃”的口号。

临时政府继续执行反人民的对外政策,并力图扑灭国内的革命火焰。临时政府不顾人民的死活,继续进行世界大战,激起了人民的强烈反对。7月,俄军在前线向德军发动进攻遭到失败。这一消息传到彼得格勒,首都50万工人,士兵举行了示威游行。临时政府出动军队屠杀和平示威者,开枪打死打伤七百多人。接着,临时政府开始大肆逮捕和杀害布尔什维克党人。布尔什维克党被迫转入地下状态。

由于形势发生了变化,布尔什维克党在8月确定了武装起义的方针。11月6日晚,列宁秘密来到彼得格勒的起义总指挥部——斯莫尔尼宫,领导起义。11月7日,彼得格勒起义取得胜利。

11月6日晚到7日清晨,20万起义工人和革命士兵迅速占领了彼得格勒的主要桥梁,火车站,邮电局,国家银行和政府机关,只剩下临时政府的所在地冬宫还未攻下。7日晚,以“阿芙乐尔”好巡洋舰的炮声为信号,工人,士兵向冬宫发起了猛烈进攻。次日清晨,起义者占领冬宫,临时政府的大部分成员被捕。

两个政权并存(临时政府:民主立宪派)

布尔维克原本支持社会革命党,1917年4月列宁回到俄国首都彼得格勒后,列宁发表《论无产阶级在这次革命中的任务》(四月提纲)。提出从资产阶级民主革命过渡到社会主义革命的路线,全部政权归苏维埃。

七月事件(俄历7月3~4日),八月叛乱(俄历8月25~9月1)

1.临时政府宣布俄罗斯为共和国,克伦斯基为总理;2.布尔什维克开始掌握彼得格勒苏维埃的领导权。

十月革命的胜利

10月23日,布尔什维克党中央委员会通过了进行武装起义的决议;
11月4日,布尔什维克苏维埃控制了首都卫戍部队;
11月7~8日,工人赤卫队和革命士兵攻克冬宫,临时政府被推翻,革命取得胜利;

《告工人士兵和农民书》,宣告各地全部政权一律归工兵苏维埃;

《和平法令》和《土地法令》;

彼得格勒武装起义胜利后,俄国建立了世界上第一个工人士兵苏维埃政府——任命委员会,列宁当选为主席。苏维埃政权随即在全国范围内建立起来。措施:建立新型的无产阶级政权,将银行,铁路和大工业企业收归国有;颁布《土地法令》,没收地主的寺院的土地,分配给农民耕种;同德国,奥匈帝国议和,退出第一次世界大战。1918年3月,首都从彼得格勒迁到莫斯科。然而,国内外的反动势力联合起来,企图将新生的苏维埃政权扼杀在摇篮中。经过三年艰苦的国内战争,1920年,苏维埃粉碎了外国的武装干涉和国内的反革命叛乱,取得了国内战争的胜利,巩固了世界上第一个无产阶级国家。

十月革命胜利后,帝国主义国家对苏俄发动了武装干涉。俄国的地主,资本家和沙皇军官也掀起了叛乱。国内战争时期,为了战胜敌人,苏维埃政权一方面加强红军的建设,一方面实行经济上高度集中的“战时共产主义”政策,主要措施是实行余粮收集制,征集农民手中的粮食和其他农产品;对中小型企业实行国有化;取消自有贸易,由国家集中分配一切生活必需品和食品。

十月革命胜利的意义

对世界历史进程的影响:打破资本主义一统天下的局面,社会主义由理想变为现实,开辟了人类探索社会主义发展道路的新时代;

对俄国历史进程的影响:俄国历史上最深刻的一次社会革命,建立了世界上第一个社会主义国家,并尝试用社会主义方式改造俄国。

\subsection{巴黎和会}
除战败国及苏俄以外,均派代表参加。四巨头:意大利首相奥兰多,美国总统威尔逊,英国首相劳合·乔治,法国总理克里蒙梭,以后三者为主开展巴黎和会。

内容包括瓜分德国的全部海外殖民地,将德国在中国的殖民地转接给日本等。巴黎和会的《凡尔赛条约》是在强权政治原则基础上,强压给德国身上。并且将战争的责任全部归于德国,而忽略了共同责任。

《凡尔赛条约》将德国居民置身于水深火热中,使德国的极端民族主义蔓延,间接地促成了纳粹政权的建立。

1919年6月,协约国与德国签订了《凡尔赛条约》
\begin{itemize}
    \item 领土:由法国收回阿尔萨斯和洛林;
    \item 军事:禁止德国实行义务兵役制;不许拥有空军,陆军人数不得超过10万;莱茵河东岸50千米内,德国不得设防;
    \item 赔款:由协约国设立“赔款委员会”,决定德国战争赔款总数;
    \item 殖民地;德国的全部殖民地,由英,法,日等国瓜分。
\end{itemize}

1912~1920年,协约国还分别同德国的盟国奥地利,匈牙利,土耳其,保加利亚签订了一系列和约,这些和约同《凡尔赛合约》一起,构成了凡尔赛体系,确立了帝国主义在欧洲,西亚,非洲统治的新秩序。

根据和约规定,1920年1月成立了国家联盟。由于美国没有加入,所以它被英法控制。

对《凡尔赛和约》的评价:

法国元帅福熙:“这不是和平,这是20年的休战!”

列宁:“靠《凡尔赛和约》来维持的整个国家体系,国际秩序是建立在火山口上的。”

\subsection{凡尔赛-华盛顿体系}
美国对《凡尔赛条约》不满,要求进行华盛顿会议。

对亚太地区的“安排”:巴黎和会虽然暂时调整了帝国主义国家在西方的关系,但它们在东亚,太平洋地区的矛盾仍然十分尖锐,日美之间的矛盾尤为激烈。在美国的倡议下,1921~1922年,美,英,法,日,意,荷,比,葡和中国九国代表在华盛顿举行会议。在华盛顿会议上起主要作用的是美,英,日三国。

国际联盟的确立

当时的国际环境要求建立一个新的国际体系,一个具有政治约束力的常设机构,即建立一个由主权国家参加的政治性组织。

具有弊端:具有普遍性原则,无法有效制止战争发生;同时退出国际联盟的国家也使普遍性不完整。

华盛顿主要针对各国家在远东及太平洋地区的冲突,签署了《五国海军协定》,从而限制了英国与日本的海军力量。并且签署了《九国公约》,在亚太地区形成了新的政治军事格局。

凡尔赛——华盛顿体系承认了列强相对实力的变化。

\subsection{二十年代的和平与繁荣}
法国与德国针对赔款问题,而产生鲁尔危机。主要是因为法国为了逼迫德国支付战争赔款,法国联合比利时出兵占领德国鲁尔地区。

德国通货膨胀严重,人民生活艰难。成立国际赔偿委员会(道威斯计划),向德国提供大量贷款,以复兴德国,这是对德国态度的转折点。

关于欧洲安全保障问题,成立《洛加诺公约》1925年,后又签订了《非战公约》,放弃以战争形式实现国家政策的方式,体现了人类的进步。在这次会议上,英,法,日,美签订了《四国条约》,相约尊重彼此在太平洋属地的权益。接着,美,英,日,法,意签订了《五国条约》,规定五国海军主力舰的吨位比例为5:5:3:1.75:1.75。

1922年,九国代表签署了关于中国问题的《九国公约》,这个公约宣称尊重中国的主权,独立与领土的完整,遵守各国在中国的“门户开放”,“机会均等”的原则。这实际上为美国在中国的扩张提供了方便。

毛泽东:“华盛顿会议使中国“回复到几个帝国主义国家共同支配的局面。””

华盛顿会议是凡尔赛会议的继续,它确立了帝国主义在东亚,太平洋地区的统治秩序。通过这两次国家会议,帝国主义列强建立了“凡尔赛-华盛顿体系。”但这一体系不可能消除帝国主义国家之间的矛盾,因此不可能长期维持下去。

列宁主张同资本主义国家和平共处,签订《拉巴洛条约》,与德国和平共处。

在20世纪20年代,特别是在1924~1929年间,资本主义世界基本上处于相对稳定时期,主要资本主义国家的经济“繁荣”一时。生产经济大幅发展,汽车进入家庭,各国经济联系密切。美国则迎来了“柯立芝繁荣”,蕴含了资本主义危机。1929年爆发,迅速席卷了整个资本主义国家。


\subsection{世界性经济危机及各国的应对}
\subsection{1929年世界性经济危机}
背景:19世纪20年代资本主义国家的经济繁荣并没有带来共同富裕,反而加大了贫富差距,引起了经济发展的不平衡。消费具有很大的盲目性,过度的分期付款为危机埋下隐患,反应了生产能力与消费能力的不平衡。股票价格的变化增加了股票市场的不稳定性。美国建立的以美元为中心的金融体系十分脆弱。

当时,在资本主义制度下,生产处于无政府状态。垄断资产阶级盲目扩大生产。20年代的美国,国民生产总值创历史新高,汽车工业,电气工业,钢铁工业和建筑业都出现了高涨局面。美国的汽车数量在十几年里,几乎翻了三番;工厂的电气化程度大大增加。20年代的美国,收音机相当普及。冰箱,洗衣机,吸尘器,电话等开始进入家庭,有声电影也问世了。
由于人民的消费能力没有相应提高,产品销售不出去,引起产品积压。美国政府和资本家为鼓励人民消费,使用了大量刺激手法,如有意压低贷款利息,以分期付款等方式刺激超前消费;公司老板拼命促使投资者购买自己公司的股票。

这次危机爆发前,美国的汽车工业和建筑业已经停滞不前,有人已经预感到生产过剩危机的临近。1929年10月24日,纽约股票交易所里突然掀起了抛售股票的狂潮这一天被称为“黑色星期四”。几天之内,股票价格暴跌。一场巨大的危机就从这里开始,迅速蔓延全美国,进而席卷整个资本主义世界。

特点:
1.涉及范围特别广:影响到整个资本主义世界,造成农业,商业和金融部门的危机;

2.持续时间比较长:1929到1933,前后5个年头;
3.破坏性特别大;1933与1929相比,整个资本主义世界工业生产下降了三分之一以上,资本主义世界的贸易额缩减了三分之二。

起点;1929年纽约证券交易所股票价格急剧下跌。

金融业的崩溃继而导致生产下降,失业增加,造成银行挤兑,使美国银行损失惨重,同时农业大批破产。美国的经济危机通过国际贸易交流致使危机迅速传播,致使世界贸易额急剧缩减。与此同时,垄断资本家毫无怜悯之心,为保持商品价格,维持利润,宁愿大量销毁产品。

\subsection{罗斯福新政}
胡佛坚持自有放任主义,致使危机不断加深。

1933年罗斯福就任美国总统

罗斯福新政的内容:1.从改革银行开始,银行放假。“炉边谈话”,罗斯福劝说国民重新将钱存入银行,并成立联邦存储保障中心;
实行美元部分贬值,促进贸易;3.通过《农业调整法》,控制基本农产品的产量,提高农产品价格从而保障农民利益;4.工业上通过《全国工业复兴法》,成立全国复兴署,工人有组织工会权力,并成立劳工关系委员会调整雇主与工人关系;5.以工代赈,在全国兴建工程项目;6.通过《社会保险法》,建立福利保障制度。恢复了人民对于国家制度的信心。

实行新政的目的,是在资本主义制度内部进行调整,加强国家对经济的干预与指导,以消除经济危机。主要措施:新政的中心措施是对工业的调整。根据《国家复兴法》,各工业企业指定本行业的公平经营规章,确定各企业的生产规模,价格水平,市场分配,工资标准和工作日时数等,以防止出现盲目竞争引起的生产过剩,从而加强了政府对资本主义工业生产的控制与调节。

罗斯福大力整顿银行,迅速恢复银行的信用,使私人现款又存入银行,重新流通。在农业方面,让农民缩减大片耕地,屠宰大批牲畜,由政府付款补贴。

危机期间,政府大力兴建公共工程,如筑路,架桥和植树。为控制一些河流及其支流的泛滥,政府拨款兴建大型水利工程,包括许多巨型堤坝和水库。这些工程有利于农业生产,便利了水上交通,吸纳了大量的失业者,使美国政府获一举多得之利。

意义:新政取得了显著效果,美国经济缓慢地恢复过来,人民的生活得到改善;资本主义制度得到调整,巩固与发展;资本主义国家对经济的宏观控制与管理得到加强;美国联邦政府的权利明显加强。新政在美国和世界资本主义发展史上具有重要意义。

\subsection{英国、法国的危机与应对}
英国的经济危机相对缓和,对工业冲击小,对农业打击严重,对外贸打击严重;(重组内阁,新政府颁布新的《失业法》)

措施:改变以往扩大外贸方式,采取提高关税政策,保护国内市场,保持了资产阶级代议制。

法国:勃鲁姆进行改革

\subsection{意大利法西斯统治的确立}
一战后,意大利经济衰退,政治混乱,工农运动高涨,中央政府几乎瘫痪。墨索里尼乘机组织法西斯党,独裁统治应运而生。1922年,法西斯党徒向首都罗马进军,法西斯专政在意大利建立起来。墨索里尼对内实行独裁统治,对外醉心于扩张,梦想恢复古代罗马帝国的疆界,地位和威严。

倡导者:墨索里尼  发动政变,夺取国家政权,取缔所有其他政党,控制新闻舆论,取消工会,干预个人生活。同时政府大力推动发展工业,保护本国工业,提高关税。

\subsection{纳粹德国的兴起}
背景:20世纪 二三十年代的世纪经济危机,也严重打击了德国。1932年,德国的工业生产下降了40\%多,失业人数超过600万。受工业危机影响,银行纷纷倒闭,对外贸易额锐减。经济危机激化了社会阶级矛盾。以希特勒为首的法西斯组织纳粹党,利用德国社会各阶层对政府的普遍不满,趁势发展壮大。

在整个德国陷于绝望时,他们不失时机地展开欺骗宣传活动,从美好的许诺,赢得许多中下层人民的信任,获得了统治阶级和大垄断资本家在政治上和经济上的大力支持。

纳粹党巧妙利用德国人民痛恨《凡尔赛条约》,渴望民族复兴的心理,煽动复仇主义情绪和种族狂热。鼓吹日耳曼人必须以“战争获得生存空间”,要“用德国的剑为德国的犁取得土地”。纳粹党由此获得了更广泛的支持。1932年成为国会的第一大党。

在政治危机十分严重的情况下,1933年希特勒上台,逐渐集总统和总理大权于一身,称为国家元首。希特勒一上台,就着手建立法西斯恐怖独裁统治。世界大战的欧洲策源地形成。纳粹党利用“国会纵火案”,打击德国共产党,逮捕和迫害大批共产党人和进步人士。

接着,纳粹党趁机解散了一切工会,取缔了除纳粹党以外的所有政党。希特勒还强化专政机器,镇压,迫害革命者和反法西斯战士。纳粹政权为加强思想控制,焚毁大量进步书籍,妄图毁灭人类先进的思想和文化等。并且掀起了迫害犹太人的狂潮。

在道威斯计划下获得美国大量贷款,经济上严重依赖其它国家。同时经济危机引发了政治危机。

阿道夫·希特勒,反对民主,呼吁独裁,1923年曾发动“啤酒馆暴动”,企图夺取政权。

德国的议会民主制原本正常运转,纳粹党占有很少数席位。但在经济危机情况下,人们对现有政府不满,同时希特勒大肆发表演讲,并积极寻求大资本家的支持,并且在国民选举中最终获胜。

希特勒上台后因授权法案而获得独裁统治,并不断取缔其它党派,工会,控制电台,新闻等,以巩固自己的统治,集党,政,军于一身。

在国内恢复了就业,加强了国家的经济职能,并对企业加强了管理而非美国的指导。对经济的干预主要为军事经济服务,并且修建军事工程和高速公路等,来吸引劳工就业,为侵略做好准备。

\subsection{日本法西斯统治的确立}
日本法西斯建立具有一定的历史传统,大肆鼓吹军事独裁,并且广泛建立了法西斯团体。战后日本经济发展缓慢,爆发的经济危机使失业人数激增,农业受到重创,加之日本严重依赖国外市场,使得经济迅速下滑,国内矛盾冲突不断。开始发展军事经济,垄断资本主义开始与国家结合。工业结构发生变化,开始大力发展重工业,由于广泛的国际市场的自有竞争对日本没有优势,日本转而萌生侵略想法。

日本军部是日本法西斯实力的支持者和集中地。1936年2月,日本军部内部的少壮派军官发动兵变失败。结果,军部内主张建立“高度国防国家”,加强对外侵略扩张的一派控制了政府。日本军部法西斯专政建立起来。世界大战的亚洲策源地形成了。

世界经济危机打击下的日本,也面临着严重的社会经济危机。以军部为主力的法西斯分子为摆脱危机,对进步组织残酷镇压,并积极怂恿向外侵略扩张,日本对中国东北领土觊觎已久,1931年9月发动侵华战争,并很快霸占整个中国东北。此后,日本军国主义又不不蚕食中国东北。1937年7月,日本发动了全国侵华战争。

《田中奏折》(《对华政策纲领》)“欲征服世界,必先征服中国,而欲先征服中国,必先征服满蒙。”

军部法西斯势力确定,广田弘毅确立了军部法西斯的统治地位,日本开始走向法西斯化。

\subsection{苏联的社会主义建设}

\subsubsection{列宁对社会主义建设道路的探索}
国内战争结束后,苏俄进入和平建设时期。苏维埃政权面临的首要任务是恢复被战争严重破坏的经济。在列宁的领导下,1921年苏俄开始实施新经济政策,允许多种经济并存,大力发展商品经济,促进国民经济的恢复与发展。

根据新经济政策,农民在向国家教了粮食税以后,余粮归自己支配,他们的生产积极性大大提高;除大型企业仍然由国家管理外,允许私人和外国资本家经营一些小型中小型企业;废除生活必需品的配给制,恢复自由贸易。由于采取了一系列措施,到1927年,国民经济回复到战时1913年的水平。1922年底,苏维埃社会主义共和国联盟成立,简称“苏联”。当时加入苏联的有俄罗斯联邦,外高加索联邦,乌克兰和白俄罗斯。后来,苏联扩大到15个加盟共和国。

苏联之前的经济政策:1.战时共产主义政策;2.新经济政策

主要为了加快经济建设而建立了高度集中的计划经济体制。

\subsubsection{斯大林模式}
1.社会主义工业化

1925年4月,俄共(布)召开了第十四次全国代表会议,斯大林提出苏联一国可以建成社会主义的理论;12月,俄共(布)召开第十四次代表大会,确认“实现社会主义工业化是当前工作的中心任务,并通过了社会主义工业化方针。基本内容:从农业国发展为工业国,重心是发展重工业。”

苏联工业化的特点:1.集中一切力量优先发展重工业;2.超高速的工业化;3.剥夺农民来保证工业化所需的资金(“贡税”理论)

原因:1.苏联处于资本主义国家的包围中;2.苏联经济落后,依靠重工业迅速发展经济;3.斯大林认为从轻工业开始发展过于缓慢。

五年计划的影响:1.五年计划的实施使苏联成为一个现代化工业强国;2.为后来粉碎法西斯侵略奠定了物质基础;3.在国际上产生了巨大反响;4.但过分重视发展重工业,农业和轻工业长期落后,影响国民经济的持续增长。
2.农业全盘集体化运动

背景:1.1927年,联共十五大确定了农业集体化的方针;2.1928年发生粮食收购危机。为了适应工业化迅速发展的需要,自30年代初起,苏联开始进行大规模的农业集体化运动。在农业集体化运动中,逐渐放弃了新经济政策时期的做法。为了加速实现农工集体化,强迫农民加入集体农庄,一些地方甚至把农民的家畜,家禽等完全收归国有。富农阶级成为集体化过程中的打击,消灭对象。到1937年,苏联全国完成了农业集体化。3.1928年到1937年,苏联先后完成了第一,第二个五年计划,重点发展重工业。这两个五年计划完成后,苏联由传统的农业国变成了强盛的工业国,国防力量也大为加强。与此同时,苏联加快了实现农业集体化的步伐。

1930年1月初,联共(布)中央政治局通过了“关于集体化的速度和国家帮助集体农庄建设的办法”的决议,决定将全国分成三个区域,在第一个五年计划期间,基本完成农业集体化。

在全国基本实现社会主义工业化和农业集体化的基础上,苏联在1936年通过新宪法,宣布苏联是“工农社会主义国家”。新宪法的制定,标志着苏联高度集中的经济政治体制的形成。这一体制也被称为“斯大林模式”。

高度集中的经济政治体制,在经济方面的特点是,国家用指令性计划管理一切经济活动;限制商品货币关系,否认市场的作用;用剥夺农民和限制居民提高生活水平的做法,实现高积累,多投资,片面发展重工业。在政治方面的特点是权利高度集中;忽视民主法制建设,各级领导实际上由上级指派,基本不受群众监督;权力越来越集中在少数人手里。

斯大林模式是在苏联外有帝国主义包围,国内经济,文化相对落后的情况下形成的,它在一定历史阶段里发挥过积极作用。但这一模式有严重 的弊端:

\begin{itemize}
    \item 优先发展重工业,使农业和轻工业长期处于落后状态;
    \item 在计划经济体制下,片面强调产值和产量,造成了产品品种少,质量差;
    \item 国家从农民手中拿走的东西太多,严重地损害了广大农民的利益,农民没有生产积极性,农业长期停滞不前;
    \item 经济发展粗放,经济效益低下,大量消耗和浪费了资源;
\end{itemize}

1930年又开始实施消灭富农的政策。苏联集体化运动中的进程曲折,农牧业的产量明显下降。1931年和1932年,连续发生大饥荒。
斯大林政策阻碍了苏联的经济发展,影响了其它社会主义国家。






