\section{第一次世界大战及战后国家格局}

\subsection{第一次世界大战的爆发}
背景:第二次工业革命引起大国的经济政治发展不平衡,实力对比发生变化,新兴资本主义国家强烈要求按照实力重新瓜分世界。帝国主义国家之间的矛盾十分尖锐。后起的资本主义国家,如德国,要求重新分割世界,与英,法,俄等老牌资本主义国家展开了激烈的争夺。争夺霸权的结果,形成了两大敌对的帝国主义侵略集团——德国,奥匈帝国,意大利组成的三国联盟和英,法,俄组成的三国协约。这两大军事集团展开疯狂的扩军备战,世界大战一触即发。

德国在俾斯麦下台后,改变了原有的大陆政策,开始了新的“全世界”政策,即与其它国家争夺海外殖民地。

同时各国之间存在的矛盾与利益很快促成了国家之间的联盟,即德奥意大利为代表同盟国,英法俄为代表的协约国。

导火索:奥匈帝国皇储斐迪南大公夫妇在波斯尼亚首都萨拉热窝被暗杀。各国并没有采取缓和措施来化解矛盾,反而更激化了矛盾,产生了局部战争。更加深了帝国主义的战斗情绪。1914年7月底,奥匈帝国向塞尔维亚宣战。第一次世界大战爆发。德国与奥匈帝国为了加强同俄国,法国的对抗,1879年缔结了军事同盟条约。1882年,意大利同法国的矛盾加剧,也加入了德奥军事同盟,德奥意三国同盟形成。三国同盟的核心是德国。20世纪初,英国与德国的矛盾激化,兴国协调了与法,俄的关系,在1904年和1907年分别与法国和俄国签订协约,组后形成了三国协约。

各国宣战:德国于8月1日和3日分别对俄,对法宣战;8月4日英国对德宣战;8月6日奥匈帝国对俄国宣战。由于英日同盟,日本也对德宣战。以欧洲为战场的第一次世界大战开始了。

\subsection{第一次世界大战的经过}
为四个战线:西线,东线,巴尔干战线,意大利战线

马恩河战役中德国战败,使德国速战的想法破产;

历经10个月之久的凡尔登战役后,同盟国丧失主动权;

原本中立的美国在德国开始潜艇时加入协约国,对德宣战,进行反潜艇作战,德奥集团处于劣势。

俄国十月革命后签订《布列斯特和约》,退出战争;

签订贡比涅协定。

第一次世界大战规模空前。战场由最初的欧洲扩大到非洲,亚洲和太平洋地区,前后有30多个国家卷入战争。1917年,美国对德宣战,参加协约国;中国也加入协约国方面作战。大战末期,同盟国方面的国家逐渐支撑不住,先后投降。1918年11月,德国投降,历时4年多的第一次世界大战以同盟国的失败告终。结束了欧洲的全权霸主地位,经济惨遭破坏。建立了《凡尔赛·华盛顿》体系。

\subsection{俄国十月革命}
背景:沙俄帝国矛盾丛生,危机四伏。加上第一次世界大战对俄国的影响。社会矛盾引起工人起义,士兵起义各方面结合,最终爆发二月革命,结束了沙皇体制(二月革命(俄历2月23~3月2日)),沙皇退位,俄国建立了资产阶级临时政府,与二月革命中建立的工人士兵代表苏维埃同时存在,临时政府掌握着主要权力。

“苏维埃”是俄语cobet的音译,是“会议”或“代表会议”的意思。俄国二月革命中,工人和士兵建立了新的革命政权——工人士兵代表苏维埃。二月革命后,俄国建立了资产阶级临时政府,形成了两个政权并存的局面。临时政府企图消灭苏维埃建立自己的“一统天下”。列宁既是提出了“一切权利归苏维埃”的口号。

临时政府继续执行反人民的对外政策,并力图扑灭国内的革命火焰。临时政府不顾人民的死活,继续进行世界大战,激起了人民的强烈反对。7月,俄军在前线向德军发动进攻遭到失败。这一消息传到彼得格勒,首都50万工人,士兵举行了示威游行。临时政府出动军队屠杀和平示威者,开枪打死打伤七百多人。接着,临时政府开始大肆逮捕和杀害布尔什维克党人。布尔什维克党被迫转入地下状态。

由于形势发生了变化,布尔什维克党在8月确定了武装起义的方针。11月6日晚,列宁秘密来到彼得格勒的起义总指挥部——斯莫尔尼宫,领导起义。11月7日,彼得格勒起义取得胜利。

11月6日晚到7日清晨,20万起义工人和革命士兵迅速占领了彼得格勒的主要桥梁,火车站,邮电局,国家银行和政府机关,只剩下临时政府的所在地冬宫还未攻下。7日晚,以“阿芙乐尔”好巡洋舰的炮声为信号,工人,士兵向冬宫发起了猛烈进攻。次日清晨,起义者占领冬宫,临时政府的大部分成员被捕。

两个政权并存(临时政府:民主立宪派)

布尔维克原本支持社会革命党,1917年4月列宁回到俄国首都彼得格勒后,列宁发表《论无产阶级在这次革命中的任务》(四月提纲)。提出从资产阶级民主革命过渡到社会主义革命的路线,全部政权归苏维埃。

七月事件(俄历7月3~4日),八月叛乱(俄历8月25~9月1)

1.临时政府宣布俄罗斯为共和国,克伦斯基为总理;2.布尔什维克开始掌握彼得格勒苏维埃的领导权。

十月革命的胜利

10月23日,布尔什维克党中央委员会通过了进行武装起义的决议;
11月4日,布尔什维克苏维埃控制了首都卫戍部队;
11月7~8日,工人赤卫队和革命士兵攻克冬宫,临时政府被推翻,革命取得胜利;

《告工人士兵和农民书》,宣告各地全部政权一律归工兵苏维埃;

《和平法令》和《土地法令》;

彼得格勒武装起义胜利后,俄国建立了世界上第一个工人士兵苏维埃政府——任命委员会,列宁当选为主席。苏维埃政权随即在全国范围内建立起来。措施:建立新型的无产阶级政权,将银行,铁路和大工业企业收归国有;颁布《土地法令》,没收地主的寺院的土地,分配给农民耕种;同德国,奥匈帝国议和,退出第一次世界大战。1918年3月,首都从彼得格勒迁到莫斯科。然而,国内外的反动势力联合起来,企图将新生的苏维埃政权扼杀在摇篮中。经过三年艰苦的国内战争,1920年,苏维埃粉碎了外国的武装干涉和国内的反革命叛乱,取得了国内战争的胜利,巩固了世界上第一个无产阶级国家。

十月革命胜利后,帝国主义国家对苏俄发动了武装干涉。俄国的地主,资本家和沙皇军官也掀起了叛乱。国内战争时期,为了战胜敌人,苏维埃政权一方面加强红军的建设,一方面实行经济上高度集中的“战时共产主义”政策,主要措施是实行余粮收集制,征集农民手中的粮食和其他农产品;对中小型企业实行国有化;取消自有贸易,由国家集中分配一切生活必需品和食品。

十月革命胜利的意义

对世界历史进程的影响:打破资本主义一统天下的局面,社会主义由理想变为现实,开辟了人类探索社会主义发展道路的新时代;

对俄国历史进程的影响:俄国历史上最深刻的一次社会革命,建立了世界上第一个社会主义国家,并尝试用社会主义方式改造俄国。

\subsection{巴黎和会}
除战败国及苏俄以外,均派代表参加。四巨头:意大利首相奥兰多,美国总统威尔逊,英国首相劳合·乔治,法国总理克里蒙梭,以后三者为主开展巴黎和会。

内容包括瓜分德国的全部海外殖民地,将德国在中国的殖民地转接给日本等。巴黎和会的《凡尔赛条约》是在强权政治原则基础上,强压给德国身上。并且将战争的责任全部归于德国,而忽略了共同责任。

《凡尔赛条约》将德国居民置身于水深火热中,使德国的极端民族主义蔓延,间接地促成了纳粹政权的建立。

1919年6月,协约国与德国签订了《凡尔赛条约》
\begin{itemize}
    \item 领土:由法国收回阿尔萨斯和洛林;
    \item 军事:禁止德国实行义务兵役制;不许拥有空军,陆军人数不得超过10万;莱茵河东岸50千米内,德国不得设防;
    \item 赔款:由协约国设立“赔款委员会”,决定德国战争赔款总数;
    \item 殖民地;德国的全部殖民地,由英,法,日等国瓜分。
\end{itemize}

1912~1920年,协约国还分别同德国的盟国奥地利,匈牙利,土耳其,保加利亚签订了一系列和约,这些和约同《凡尔赛合约》一起,构成了凡尔赛体系,确立了帝国主义在欧洲,西亚,非洲统治的新秩序。

根据和约规定,1920年1月成立了国家联盟。由于美国没有加入,所以它被英法控制。

对《凡尔赛和约》的评价:

法国元帅福熙:“这不是和平,这是20年的休战!”

列宁:“靠《凡尔赛和约》来维持的整个国家体系,国际秩序是建立在火山口上的。”

\subsection{凡尔赛-华盛顿体系}
美国对《凡尔赛条约》不满,要求进行华盛顿会议。

对亚太地区的“安排”:巴黎和会虽然暂时调整了帝国主义国家在西方的关系,但它们在东亚,太平洋地区的矛盾仍然十分尖锐,日美之间的矛盾尤为激烈。在美国的倡议下,1921~1922年,美,英,法,日,意,荷,比,葡和中国九国代表在华盛顿举行会议。在华盛顿会议上起主要作用的是美,英,日三国。

国际联盟的确立

当时的国际环境要求建立一个新的国际体系,一个具有政治约束力的常设机构,即建立一个由主权国家参加的政治性组织。

具有弊端:具有普遍性原则,无法有效制止战争发生;同时退出国际联盟的国家也使普遍性不完整。

华盛顿主要针对各国家在远东及太平洋地区的冲突,签署了《五国海军协定》,从而限制了英国与日本的海军力量。并且签署了《九国公约》,在亚太地区形成了新的政治军事格局。

凡尔赛——华盛顿体系承认了列强相对实力的变化。

\subsection{二十年代的和平与繁荣}
法国与德国针对赔款问题,而产生鲁尔危机。主要是因为法国为了逼迫德国支付战争赔款,法国联合比利时出兵占领德国鲁尔地区。

德国通货膨胀严重,人民生活艰难。成立国际赔偿委员会(道威斯计划),向德国提供大量贷款,以复兴德国,这是对德国态度的转折点。

关于欧洲安全保障问题,成立《洛加诺公约》1925年,后又签订了《非战公约》,放弃以战争形式实现国家政策的方式,体现了人类的进步。在这次会议上,英,法,日,美签订了《四国条约》,相约尊重彼此在太平洋属地的权益。接着,美,英,日,法,意签订了《五国条约》,规定五国海军主力舰的吨位比例为5:5:3:1.75:1.75。

1922年,九国代表签署了关于中国问题的《九国公约》,这个公约宣称尊重中国的主权,独立与领土的完整,遵守各国在中国的“门户开放”,“机会均等”的原则。这实际上为美国在中国的扩张提供了方便。

毛泽东:“华盛顿会议使中国“回复到几个帝国主义国家共同支配的局面。””

华盛顿会议是凡尔赛会议的继续,它确立了帝国主义在东亚,太平洋地区的统治秩序。通过这两次国家会议,帝国主义列强建立了“凡尔赛-华盛顿体系。”但这一体系不可能消除帝国主义国家之间的矛盾,因此不可能长期维持下去。

列宁主张同资本主义国家和平共处,签订《拉巴洛条约》,与德国和平共处。

在20世纪20年代,特别是在1924~1929年间,资本主义世界基本上处于相对稳定时期,主要资本主义国家的经济“繁荣”一时。生产经济大幅发展,汽车进入家庭,各国经济联系密切。美国则迎来了“柯立芝繁荣”,蕴含了资本主义危机。1929年爆发,迅速席卷了整个资本主义国家。


