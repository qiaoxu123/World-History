\section{近代世界的孕育}

\subsection{文艺复兴}
\subsubsection{背景}
文艺复兴发生主要是来自于罗马天主教的困境。14世纪中半期,欧洲爆发黑死病,是欧洲历史上规模最大,危害最惨烈的一次。三分之一以上人口死亡,加之经济萧条,战争等因素,人们不断地将精神需求投向教会,但罗马天主教教会却一次又一次陷入丑闻中,教皇克雷芒五世对法王腓力四世言听计从,即所谓的“阿维农之囚”,教皇威严一落千丈。同时产生了大分裂,即基督教会历史上最大的丑闻,乌尔班六世(罗马)和克雷芒六世(阿维农)各宣称为正统,从而使基督教世界濒于分裂,而且大分裂发生在瘟疫盛行时,于是人们开始将精神世界投向宗教以外的世界。

“文艺复兴教皇”利奥十世在位时,不断扩大宗教影响,征召文人巧匠修建教廷,是意大利文艺复兴的赞助者。至此,教会控制之外的新思路不断出现。

14世纪前后,地中海区域是欧洲贸易最繁荣的地区。当时,意大利兴起了许多城市国家。这些城市国家商业发达,手工工场也发展起来,越来越多的人通过发展工商业富裕起来,开始产生新兴的资产阶级。

当时,意大利著名的工商业城市国家有威尼斯,热那亚和弗洛伦萨等。弗洛伦萨不仅有发达的银行业,呢绒业也很繁荣。呢绒业手工工场有200多家,雇佣了约3万名员工,年产呢绒多达十几万匹,行销各地。那些开设手工工场的工场主和富商就构成了新兴的资产阶级。

\subsubsection{文艺复兴}
Renaissance 原意为“rebirth”,即“再生”。就是恢复古希腊罗马文化。在恢复古典文化和艺术的同时诞生了一种新的时代精神。雅各布·布克哈特在《意大利文艺复兴时期的文化》中将文艺复兴的实质概括为“世界的发现和人的发现”。

意大利作为当时欧洲经济最繁荣的地区,建立在东方的垄断贸易基础中。佛罗伦萨是当时欧洲的纺织业中心,手工业中心。意大利的经济是文艺复兴的重要基础,为文艺复兴提供了充分的财力支持,如美地奇家族的罗伦佐,创建了柏拉图学院。同时,意大利的经济环境很容易培养出一种与中世纪基督教教诲相悖的世俗精神。意大利作为当时的城市国家,自治程度最高,多由工商业者统治,美地奇家族自由的统治阐述了一种完全不同的统治思想,既由财富和才能决定等级,即由贵而富。意大利拥有丰富的古典文化遗产,便于思想文化的传播。

\subsubsection{人文主义}
文艺复兴时期的时代精神主要表现为人文主义,人文主义有狭义和广义之分,狭义的人文主义是指人文学科,包括文法,诗歌,修辞,历史和道德哲学等。广义的人文主义是指文艺复兴时期的人生观和价值观,是关于人的学问。它主要的表现是重视现世生活,即由天堂转为人间。重视人的价值,强调人的尊严。同时追求个人主义,追求卓越,崇拜才能。

\subsubsection{文艺复兴时期的文学与艺术}
早期的文学三杰,分别是但丁,彼得拉克和薄伽丘,均来自于佛罗伦萨。但丁被誉为“旧时代的最后一位诗人,新时代的最初一位诗人”,代表作是《神曲》。彼得拉克被誉为“人文主义之父”的桂冠诗人。而薄伽丘开创了欧洲近代短篇小说先河,代表作是《十日谈》。

在艺术领域,开始追求立体感,并且开始运用透视法。诞生了艺术三杰“达芬奇”“米开朗基罗”“拉斐尔”。

政治学领域有马基雅维里的《君主论》。

\subsubsection{欧洲各国的文艺复兴}
15世纪中叶,德意志的谷腾堡等人受活字印刷术的启发,造出了合金活字印刷机,研制成功了油脂性印刷油墨,还设计出了金属活字的铸字盒和冲压字模。这就是欧洲人自己研制成功的最早的印刷机。

15至16世纪,文艺复兴扩展到欧洲的其他地方。条件是造纸术和印刷术的传播,如谷登堡活字印刷术。鹿特丹的伊斯拉漠《愚人颂》,法国拉伯雷的《巨人传》,英国的莎士比亚戏剧,还有西班牙塞万提斯的《堂吉诃德》等。

\subsection{宗教改革}
主要人物是马丁·路德,反对罗马天主教的“行为称义”,主张“因信称义”。马丁·路德张贴“九十五条论纲”于维登堡大教堂,成为与罗马教皇的正面冲突的导火索。马丁·路德只是掀开了宗教改革的序幕,并未传播到欧洲。路德宗的确立“抗议者”成为新教的路德宗。

约翰·加尔文创立加尔文教,核心是“预定论”,发展了路德的新教思想,并在欧洲范围开始传播。

英国的宗教改革源于英王亨利八世的离婚案,之后英国脱离了罗马教廷,并成立教派称为英国国教或安立甘教。

\subsubsection{天主教会的反宗教改革运动}
特兰托宗教会议,重申天主教教义和仪式的正确性及教皇的最高权威,禁止教徒阅读“禁书”,加强宗教裁判所的活动,严惩异端。同时天主教会内部进行了一系列整顿,以改善教会形象,提高教会的威信。并于1534年成立耶稣会,其宗旨是维护教皇威信,重振天主教会,对抗宗教改革。

\subsection{大西洋的开通}
宗教原因包括十字军东征(基督教扩张)和欧洲伊比利亚半岛的“收复失地运动”。

商业动机只要是迫切想打破威尼斯与阿拉伯的垄断。政治原因主要来自于民族国家的兴起。技术条件主要是航海技术成熟与罗盘的使用,以及火炮的使用,改变了传统的海战技术。

马克思说过:“在16世纪和17世纪,由于地理上的发现而在商业上发生的并迅速促进了商人资本发展的大革命,是促进封建生产方式向资本主义生产方式过渡的一个主要因素。”

开辟新航路的主要过程:1.葡萄牙人的向东航行;2.1487年迪亚士到达非洲“好望角”;3.1498年达·迦马到达印度;4.麦哲伦于1519~1522年完成首次环球航行;

新航路开辟以后,从欧洲到亚洲,美洲和非洲等地的交通往来日益密切,世界开始连成一个整体;欧洲大西洋沿岸工商业经济繁荣起来,促进了资本主义的产生和发展。

价格革命和商业革命是其造成的主要影响。

\subsection{西班牙的兴衰}
西班牙是最先崛起的民族国家之一,收复失地(被阿拉伯占领),导致“无敌舰队”覆灭。教训是只有建立良好的国内基础才能更好的发展。同时一个国家的外交政策要以国家利益为本。

\subsection{法国专制主义的兴起}
法国政权提倡绝对主义-君权至上。

英法进行百年战争,以收回法国被英国占领的土地,最终法国胜利。
对法国的影响包括统一的疆域,民族国家的兴起,增强了法国的民族认同感。同时国王获得了征税权,维持常备军。

\subsection{英国民族国家的形成}
英国都铎王朝时期经历了红白玫瑰战争,兰卡斯特家族与约克家族争夺王位,最终开创了都铎王朝,加强了中央集权。

英国的宗教改革起源于亨利八世,这是一场自上而下的宗教改革,1534年颁布《至尊法》;最终排除了罗马天主势力的影响,使基督教会从属于国家,教会开始服从于国王。

王权受到的限制主要来源于议会,普通法,同时缺少一支常备军。

\subsection{英国君主立宪制的确立}
英国议会形成于13世纪,分上,下两院,传统上有批准国王征税等行动 的权利。17世纪时,许多新兴资本阶级和新贵族的代表成为下院。

\subsubsection{英国资产阶级革命的背景}
新航路开辟以后,欧洲的主要商道和贸易中心从地中海区域转移到了大西洋沿岸。英国利用有利的地理位置拓展对外贸易,进行了殖民掠夺。在此期间,制呢业等工场手工业得到了很大发展,还出现了采用资本主义经营方式的牧场和农场。

\subsubsection{英国君主立宪制确立的背景}
主要矛盾在于英国议会与国王,几个朝代与议会均有很大冲突,在查理一世时期达到高潮,解散了议会,进一步加深了与议会的矛盾。议会后经过了“短期议会”和“长期议会”,进一步加深了与议会的矛盾。导火线是1640年,英王查理一世召集议会开会,希望能够筹集军费,镇压苏格兰发生的人民起义。会议期间,议员们对国王的独断专权进行了猛烈抨击,要求限制王权,掀起了英国资产阶级革命的序幕,最后国王宣布于1642年8月22日讨伐议会,内战爆发。

双方大致分为二部分,克伦威尔率领的议会军于马斯顿荒原之战后基本击溃国王军队。战争胜利后,在对待查理一世问题上,克伦威尔清洗了议会,坚决处死查理一世。1649年1月,查理一世被处死,英国成为了共和国。后于165年,克伦威尔担任护国公。

克伦威尔于1658年去世后不久,查理二世复辟,英国建立君主共和制。
在开始时期国内政治和谐,后议会又与国王产生矛盾。主要矛盾在于国王詹姆斯二世支持天主教,并且老来得子。而以议会为首的新教派反对王位继承,于是秘密发动政变,邀请玛丽与威廉当国王,史称“光荣革命”。

1688年,国王作为国家象征保留下来,权利转移到议会手中,国王日益处于统而不治的状况。1689年,英国议会通过了《权利法案》,以限制王权。《权利法案》以法律的形式对王权进行了明确制约,规定不经议会批准,国王不能征税,也不能在和平时期维持常备军;同时还规定国王既不能随意废除法律,也不能停止法律的执行。这样,君主立宪制的资产阶级统治确立起来。

18世纪初,英国议会又通过了《王位继承法》,进一步削弱了国王的权力,巩固了议会的权力。

英国资产阶级通过革命推翻了封建君主专制,确立了自己的统治地位,为发展资本主义扫清了道路,推动了世界历史进程。

\subsection{奥地利的兴起}
奥地利与普鲁士同属于神圣罗马帝国

神圣罗马帝国(1618~1648)与欧洲新教派发生了“三十年战争”,新教与法国联合起来反对哈伯恩堡王朝,也是近代欧洲历史上第一次大的国际战争。最终战败,签订了《威斯特伐利亚合约》(1648),神圣罗马帝国破裂,促进了主权国家的兴起。

奥地利兴起后收复匈牙利领土,以维也纳为中心。推行天主教,废除了选举产生的君主政体,改为君主世袭政体。

\subsection{普鲁士的兴起}
在神圣罗马帝国境内形成,是一个军国主义国家。

兴起的原因主要有地缘环境原因和国内环境原因。在地缘环境上,普鲁士地处平原,军事地理弱,只能靠军事力量来弥补。在国内环境原因上,30年战争引起的恐慌使人民迫切希望强化军队。

表现在普鲁士拥有宏大的军队,特别是威廉一世厉行节俭,将王室开支缩减了四分之三,用于军队建设,奠定了普鲁士的强国基础。军队在国家享有崇高的地位。同时,普鲁士也建立了与军事国家相称的经济体制,并且增加税收,实行重商主义,促进经济发展,同时接纳外国移民,向贵族收税。

普鲁士的军国主义,促进了中央集权国家的形成,确立了国王对容克,即大土地贵族的征税权。加强了容克对农民的剥削,农奴制盛行。同时,中产阶级力量薄弱,无权购买土地。

最终,西里西亚战争确立了普鲁士强国地位。

\subsection{俄国的“西化”}
俄国一直以来游离于欧洲发展主流之外。主要在于以下几个原因:一是东正教教会的影响,与欧洲的天主教呈对立状态;二是蒙古人的影响,统治俄国达两个世纪之久;三是俄国缺乏不冻港,没有通向欧洲的港口。

\subsubsection{彼得大帝改革}
在彼得大帝统治时期,进行了一系列措施和行动进行改革,逐步将俄国引向西方。派使团去西欧考察,向西方学习先进制度和技术。发动了北方战争,包括纳瓦尔战役,战败。又发动了波尔塔瓦战役,最终战胜,从瑞典手中夺得了不冻港,开启了通向西方的窗口。彼得大帝从军队开始,取缔射击军,建立新式军队,仿效欧洲。建立新都,1705~1725的圣彼得堡成为新俄国的象征。同时推行重商主义改革,建立船队发展呢贸易,并且将近代工业体制嫁接到农奴制上,并且加重税收,从而限制了农奴流动。结果,俄国逐步跻身欧洲强国军队。

\subsection{科学革命}
在欧洲发展过程中,因对罗马教廷的不信任和迷信,对巫女产生迫害。
科学革命从哥白尼开始到牛顿结束。

哥白尼打破了古希腊的地心说,即“托勒密”体系,自然科学开始发展。布罗诺发展了日心说,宇宙是无限的。开普勒发现了行星运动的三大规则。
伽利略是近代实验科学开拓者,到牛顿发现万有引力定律时,科学革命发展到鼎盛时期。

欧洲也创立了许多的科学社团,包括英国皇家学会,法兰西学院,德国柏林学院。

并且逐步发展起了怀疑和批判的科学精神。

\subsection{启蒙运动}
起源于18世纪,主张运用理性去发现人。

特点是崇尚理性,如(康德),崇尚进步;崇尚强烈的批判精神(摆脱宗教迷信)。

启蒙思想家包括英国政治思想家托马斯·霍布斯和约翰·洛克,法国《百科全书》主编狄德罗,以及“法国三杰”——孟德斯鸠,伏尔泰和卢梭。十八世纪的法国是欧洲启蒙思想的战场。孟德斯鸠《论法的精神》,主张三权分立。伏尔泰主张开明君主制,如普鲁士的腓特烈大帝和俄国的叶卡特琳娜女皇。







