\section{西欧封建社会的形成与发展}
\subsection{前言}
从476年西罗马帝国灭亡,到1500年左右的欧洲历史,被西方史学家称为“中世纪”。它是古代希腊罗马文明消亡与近代资本主义社会产生之间的一段历史。中国史学家习惯上称之为西欧封建社会。

\subsubsection{西欧主要封建国家形成与发展}

西罗马帝国灭亡后,日耳曼人在其领土上建立起几个国家。其中的法兰克王国,发展为西欧的一个大国。在查理统治时期,法兰克王国达到全盛。800年,建立了查理曼帝国。814年,查理去世,他的子孙庸碌无能,相互倾轧,国家陷入混乱。843年,他们将帝国一分为三。后来,从这三部分,分别发展为法兰西,德意志和意大利三个封建国家。5世纪中期,在日耳曼人迁徙浪潮中,他们当中的盎格鲁,撒克逊等部落从欧洲大陆进入不列颠,建立了一些小的国家。这些国家长期兼并,9世纪,开始形成早期的英吉利王国。

但是,在这些秀国家中,政治的发展曲折艰难,国家长期四分五裂,统一的中央集权国家很晚才出现。直到15世纪,英国和法国菜通过强化王权,剪除大封建主实力的措施,建立起统一的中央集权国家。德意志的情况更加糟糕。从查理曼帝国分裂出来的东法兰西,逐渐发展为德意志国家。德意志国王奥托一世,野心勃勃,大举向意大利扩张,在教皇的支持下,在10世纪建立起所谓的“神圣罗马帝国”。然而,由于皇帝权利受教皇和封建主的掣肘,帝国徒有虚名。此后的几百年间,帝国内的诸侯势力强大,根本不把皇帝放在眼里。到14世纪时,帝国境内还存在七个诸侯国,处于诸侯割据的状态。在很长的历史时间里,德意志境内关卡林立,交通不便,封建混战频繁。直到19世纪中期,德意志仍存在几十个小邦,处于四分五裂的状态,统一的中央集权国家迟迟建立不起来。19世纪六七十年代,其中最强大的普鲁士才统一了德国,建立了封建性很强的军事专制统治。意大利的情况与德意志基本相当,长期处于封建分裂状态中。

西欧封建社会时期,教皇与教会不仅是西欧最大的土地所有者,还是西欧封建制度的精神支柱。他们加紧对人民的精神统治,残酷压制与教会观点相悖的“异端”思想。在精神和文明领域,神权凌驾于一切。当时的百姓多是文盲,教会完全垄断了对基督教经典《圣经》的解释权。任何背离教会的说教和反对罗马教廷教义的思想都被当做“异端”。13世纪,教会建立起“宗教裁判所”,对“罪名”成立的“异端”分子实行判决,轻者罚款,重者监禁,有的甚至被捆在火刑柱上烧死。

\subsubsection{西欧封建社会的政治——体系严密的封建制度}
西欧封建社会早期,土地是主要财富。国王把一部分土地分封给分封给大封建主,这些人成为诸侯,诸侯又把一部分土地分封给较小的封建主,小的封建主再向下分封。国王和大封建主又各自分封了一批骑士,作为自己的武装力量。

这样层层分封,就形成了各种自大到小不同等级的封建主。这些封建主分别领有大小不等的封地,拥有数量不等的庄园,农奴和武装。这样,在社会的每个角落,形成一种领主(封主)和附庸(封臣)的关系,他们彼此负有义务。领主保护附庸,附庸必须向领主宣誓效用,为领主提供多种服务,包括:发生战争时军事上要援助领主;领主到访时根据礼仪进行接待;领主的女儿结婚时送礼致贺。但是,领主只能管辖自己的附庸,不能管辖附庸的附庸。每一个封建主无异于一个小的国军,割据一方,各自为政。诸侯的势力很大,有的竟敢向国王挑战。封建贵族住在戒备森严的城堡里,有自己的武装;他们的经济单位叫庄园,一般自给自足,不与外界交往。结果,严格的封建等级制度,辽阔庄园环绕的无数城堡以及直插天际的尖顶教堂,成为西欧封建社会早期的典型政治风貌和独特的社会景观。

\subsubsection{政教冲突}
从5~6世纪始,西欧的教诲势力迅速增长。罗马天主教会是最有势力的封建领主,他拥有天主教世界土地的三分之一。教诲同样按照封建的方式建立起自己的教阶制度,最高的是教皇,下面是大主教,主教等,他们各有自己的辖区。

8世纪中期,意大利中部出现教皇国。教皇既是宗教领袖,同时又是拥有世俗权力的一国之君,直接管辖的领土达四万多平方千米。国王为了使自己的统治神圣化,经常请求教皇以上帝的名义为自己加冕。查理大帝就是这样做的。这种做法加强了封建国王与教会的联系,更意为着教权凌驾于王权之上。9世纪,教皇成为西方基督教世界中的仲裁者。然而,教皇与封建君主时而相互勾结,时而明争暗斗。12~13世纪,经过长期的争斗,教皇权利达到了顶峰,教皇有权废黜君主;罗马教廷成为中欧和西欧一切宗教事务和教义问题的最高总裁机构。只是到了中世纪后期,随着西欧中央集权国家的形成与壮大,资产阶级的兴起,文艺复兴,启蒙运动和宗教改革的发展,教皇和罗马天主教会的势力才逐渐衰落下去。

\subsubsection{城市的兴起}
11~12世纪,欧洲各地的城市普遍重新兴起。中世纪时,由于人口的增加和商业的发展,除原来罗马帝国时期的老城市外,在城堡,主教堂,大修道院附近地区出现了许多新兴城镇。法国的巴黎和马赛,英国的伦敦,德意志的科隆等都是当时的著名城市。14世纪时,伦敦人口有4万,巴黎有8万。西欧城市工商业和手工业发达。随着技术的不断发展,手工业的行业分工日益精密。同一行业的人们组成行会,行会雨后春笋般地出现。

在西欧城市重新兴起和工商业迅速发展的过程中,市民阶级形成了,并从中进一步分化出手工业者,商人和银行家等等。商人和银行家成为市民阶级的上层,发展为早期的资产阶级。广大西欧城市开展了争取自治权利的斗争,并制定了自己的法律,建立了自己的武装,向封建王权和各封建主发起了挑战。一些城市里,如法国的巴黎,英国的牛津和剑桥等还建立了大学,成为著名的大学城。到14~15世纪,从西欧封建社会内部发展起来的新兴的城市中等阶级,以及与封建制度水火不容,成为西欧反封建王权的强大革命力量。

\subsubsection{封建等级代表制的出现}
13世纪早期,欧洲一些王国的君主竭力重振王权,将地方权力收归中央。英国国王约翰横征暴敛,没收教会田产,引起了贵族和教会领袖的不满。强大的封建贵族联合教士,小封建主和市民举行武装反抗。经过斗争,约翰被迫承认教会选举自由;不向封建主征收额外的捐税;承认伦敦等城市已享有的自由。结果,王权收到封建法律的约束,国王的独断专行收到限制。国王不甘心权利的削弱,挑起内战,国王战败。13世纪后半期,英国的议会制度开始萌芽,以后议会逐渐开始定型为上下两院。国王必须通过议会规定赋税,制定法律。等级代表制的封建君主政体对国王权利有所制约。14世纪初,法国也出现了以三级为代表的等级代表制,由于法国王权比较强大,三级会议限制王权的作用相对较小。

英国的等级代表制度对西方政治制度的发展影响巨大,在反对封建国王专制集权的斗争中,封建贵族,高级教士,城市商人等联合起来斗争,迫使国王坐下来,与他们商讨有关征税等关系国计民生的重大问题,使封建王权收到制约。英国的等级代表制在制约王权,建立资本主义制度的斗争中起到了重要作用,成为西方近代议会制度的起源。


\subsection{日耳曼人的入侵及其影响}
公元5世纪末,西罗马帝国灭亡,一系列日耳曼帝国崛起,西欧开始步入封建社会,这个时期的西欧经济以封建庄园制为主,政治上长期处于王权羸弱分裂割据状态。

日耳曼人为游牧部落,于公元3-5世纪,在匈奴人的驱赶下进入西罗马帝国并逐步进行了征服。同时日耳曼人属于蛮族部落,具有两个特点:1.原始社会末期的氏族部落;2.军事民主制阶段;

日耳曼人入侵的后果:

1.改绘了当时西欧的政治地图,重组了民族格局;

2.造成了巨大的社会经济破坏(由于日耳曼人文化经济水平落后,无法继承罗马的文化);

罗马的封建制因素和日耳曼人的氏族制因素结合并最终导致了西欧封建社会的形成。

日耳曼人的特点:

\begin{itemize}
    \item 没有国家观点和行政管理机构;
    \item 法律简单,神裁法;
    \item 自然经济;
    \item 日耳曼征服者在各方面不同程度上受到了罗马文化的影响;
    \item 皈依罗马天主教,反映了当时蛮族罗马化的趋向;
\end{itemize}

\subsection{法兰克王国}
日耳曼人创立的王国中存在时间最长的。后来,在法兰克王国的基础上,建立起意大利,法兰西和德意志。

\subsubsection{墨洛温王朝(481-751)}
在496年统治者克洛维正式皈依了基督教罗马教会,这种政治需要开启了日耳曼人与罗马人结合的起点(王权与政权结合)。

之后其子统治时期庸碌无能,由宫相掌管国家大事。8世纪前期,查理·马特成为宫相,进行采邑制改革,改变以前无条件赏赐贵族土地的做法,实行有条件的土地分封,得到封地的人必须为封主服兵役。其意义在于将土地以采邑的形式分封给参战的战士,条件是为中央政权服兵役。这就在封主和封臣之间建立起了明确的权利和义务关系。9世纪后期,这种封地逐渐变成了世袭领地。后来,国王以下的封建主也把土地当做采邑分封出去。逐层分封的结果,形成了不同等级的封建主:公爵、侯爵、伯爵、子爵、男爵、骑士。这封建等级制度中,每个人对其上级来说都是附庸,对其下级来说则是封建主,各级封建主依次从属,每个封建主只能管辖自己的附庸,而无权管辖自己附庸的附庸。

大封建主居住在讲究城堡里。城堡既是住宅,有是防卫措施,周围筑起高墙,墙外挖有宽阔的壕沟。城堡内备有武器,存放着充足的粮食和生活用品。

影响:1.变更了土地占有关系,促进了以土地为纽带的封建等级制的形成;2.在封建贵族内部形成了严格的等级制度。

\subsubsection{加洛林王朝(751-843)}
751年其子丕平即位,由于丕平不满宫相这一地位,为篡夺王位,与基督教会勾结。篡位成功后,为表感恩之情,两次进攻与教皇作对的人,把罗马到拉凡纳的一片土地奉献给教皇,史称“丕平献土”。教皇以上帝的名义为他加冕,这种做法加强了国王与教会的联系,使教权凌驾于王权之上,而且教皇辖地的疆域从此得到了奠定。在意大利中部出现了教皇国。十二三世纪时,教皇国势力鼎盛一时,754年得到罗马皇帝的加冕。

西欧封建社会时期,罗马教廷有至高无上的权利。在西欧长期动乱的过程中,基督教会乘机扩大势力与影响。法兰克等国君主接受了基督教,并向教会大量赐赠地产。教会本身也巧取豪夺,占有大量土地。

\subsubsection{查理曼帝国}
法兰克达到全盛时期,在此期间教廷与皇权达到空前的结合。查理大帝支持罗马教廷。

800年,教皇为查理加冕为“罗马人的皇帝”,至此,日耳曼人完成与罗马的结合。公元843年《凡尔登条约》,查理曼帝国一分为三。


\subsection{封建制度与封建庄园经济}
在法兰克帝国时期,由于王朝混乱,各领主自己招募士兵,成为当地的保护着。

\subsubsection{封建制度}
源自拉丁语“封土”,即封军赋予义务的土地,有条件地占有土地;封建制度是以土地占有权和人身关系为基础的社会制度,在这种关系中封臣从封君得到一块土地,同时向封君履行一定的义务。

两大源头:日耳曼因素,罗马因素。

\subsubsection{封臣制度}
采邑,终身占有,不是世袭;查理·马特军事改革,采邑制度普及。重装骑兵,全身盔甲;

世袭采邑;统治权(司法权)与土地所有权的结合,即领主封臣制与大土地产馈赠的结合。

西欧封建制度核心(封君封臣关系)

封建法规:规范封君与封臣关系,相互的权利与义务关系;

骑士精神:忠勇,仗义,尊重妇女,保卫基督教;(参考《唐吉坷德》)

\subsubsection{庄园制度}
地位:1.封建庄园是整个西欧封建时代农村基本的经济与社会组织;2.国王,贵族和教会都是庄园的领主。

特点:1.经济上自给自足,除盐铁外,所有都可以在庄园里自己完成;2.社会生活上自给自足(教堂,教士);3.农奴没有人身自由,农奴的义务:服劳役,捐税和使用权;4.领主有责任保护农奴;

庄园分为三部分:领主的自营地,农奴的分地和自有佃农的分地。

封建庄园经济是一种经济上自给自足,政治和宗教上大体独立的社会实体。(参考英剧《唐顿庄园》)

\subsection{西欧经济的复兴}
经历过“黑暗的中世纪”之后,西欧经济逐渐走向复兴。

\subsubsection{农业革命}
使用了中世纪的重犁,可以开发西欧湿重的黏土,增加了作物的种植面积;

谷物轮种的三田制,替代了原先的二田制,即分为冬季作物,春季作物和休耕三部分,从而每年能收获三分之二;

风磨和水磨的发明,使得劳动率提高,耕种面积扩大,农产品产量显著增加,出现了剩余。也是中世纪城市兴起于发展的基础。

\subsubsection{贸易复兴}
两大贸易区:

南部以意大利的威尼斯,热那亚和比萨等城市为主的地中海贸易区;北部贸易区以佛兰德尔的布鲁日等城市为中心的北海贸易区;

定期市集,最著名的香槟市集(长途搬运货物的交易集所)。

\subsection{中世纪城市的兴起}

\subsubsection{工商业城市的起源}
城市多兴起于交通便利,相对安全,容易获得廉价原料和销售产品的地方。海湾地区,如意大利的威尼斯;港口附近,如英国的牛津;堡垒附近,如英国的曼彻斯特。

当时的城市规模不大,一般围有城墙和供守望用的城堡。除教堂和市政厅等较大建筑外,房屋低矮,街道狭隘,拥挤,纵横交错。沿街商店,住房和作坊鳞次栉比。有些手工业者兼营农牧业,街上牛,羊,猪,鸡,鸭等随处可见。白天嘈杂喧嚣,夜晚万籁俱寂。

随着城市的发展,阶级冲突日益尖锐。西欧城市是在教会或世俗封建主的领地上产生的。随着商品经济的发展,封建主日益贪婪,对城市市民加紧剥削。

三个主要途径

\begin{itemize}
    \item 罗马古城遗址上建立新城;
    \item 军事城堡(可以保护手工业者和商人从事交易,因而逐步发展为城市);
    \item 商业中心,如威尼斯;
\end{itemize}

以上城市往往在封建领地里发展起来。10世纪开始出现作为手工业和商业中心的城市。意大利,法国,英国,德意志等都有许多著名城市。

城市可以分为:a.城市国家;b.自治城市;c.自有城市;

根据自治程度的不同来划分,部分工商业城市还具有国家职能,如征战。同时获得自治的途径有多种,如赎买,武装斗争等。

\subsubsection{城市兴起的意义}

1.促进了欧洲工商业的发展和乡村经济的商品化(如乡村原有的实物地租变为货物地租);

2.导致了农奴制的瓦解(在城邦居住可获得自由民的身份,免除债权和义务);

3.孕育了新的社会阶层——城市中产阶级(原先中世纪等级有三:教士,贵族,农民);

\subsection{十字军东征}
时间:1096-1270年

11世纪末起,在罗马教廷的倡导下,西欧教俗封建主及大商人组成联军,以向进犯“圣地”的伊斯兰教徒展开“圣战”的名义,对地中海东部地区发动了持续近200年的侵略性的军事远征。

各个阶层参加东征的原因:

\begin{itemize}
    \item 教会:收复“圣地”;
    \item 封建领主和骑士:夺取土地和财富;
    \item 普通基督教徒:宗教信仰,教会许诺的利益;
    \item 商人:扩大商业利益;
    \item 农民:摆脱困境,获得自由。从穆斯林手中夺取“耶路撒冷”,即所谓的宗教热情。
\end{itemize}

导火索:1095年,罗马教皇召开克勒芒宗教会议,号召拯救圣城。

十字军东征的过程(1086-1270)

\begin{itemize}
    \item 第1,2,3次东征的目标是耶路撒冷及其周边地区;
    \item 第4次东征的目标是拜占庭帝国;
    \item 第5,6,7次东征的目标是埃及;
    \item 第8次东征的目标是北非的突尼斯
\end{itemize}

第一次东征时攻陷了耶路撒冷,并且进行了大屠杀。之后建立的十字军国家(1.埃德萨伯国;2.安条克公国;3.耶路撒冷王国;4.特里波利波国)

第三次(1189-1192),第四次(1202-1204)教皇英诺森二世率军攻下了拜占庭首都君士坦丁堡。

东征的后果和影响:

1.侵略战争的性质;

2.消极影响:使东方的物质文明遭到了破坏,阻碍了其社会经济的发展;

3.积极影响:客观上以野蛮的方式促进了东西方文明的交流与传播;

4.教会势力的消长;

\subsection{中世纪的法国}
中世纪的法国是西欧最早形成的民族国家。法国中央集权的加强,由法王腓力二世执政(1180-1223)。利用联姻,作战等方式,从英国统治区域收复了大量失地。

腓力四世与罗马教廷的斗争以王权胜利告终,罗马教廷也迁都到阿维农,史称“阿维农之囚”,操纵教廷长达70年。

腓力四世在于罗马教廷的斗争中为争取民众支持,召开会议(三级会议)。1302年,在巴黎圣母院召开了第一次三级会议,三个等级分别是:教会上层人士,世俗贵族和市民上层代表。会议标志着法国等级君主制形成,各个封建等级分享政权。路易十四时达到绝对君主制的顶峰。

\subsection{中世纪的英国}
在中世纪时期英国的权利很大。起于1066年的诺曼征服。威廉成为国王,加快了封建化进程。威廉一世为加强王权,采取措施如下:

\begin{enumerate}
    \item 将土地作为战利品分封,威廉实际上成为了英格兰最高土地的所有者;
    \item 进行土地大清查,制订了《土地赋役调查薄》,作为国王征收赋税的依据;
    \item 保留了盎格鲁-撒克逊人的郡制,设置了郡长和郡法庭。威廉要求土地所有者先向国王效忠,再向领主效忠,加强了王权;
\end{enumerate}
	
中世纪的英国设立了普通法,基本内容是司法判决案例,受到传统习俗,罗马法和宗教法的影响,又自成体系。在盎格鲁-撒克逊人的习惯法基础上,把全国各地的习惯法集中起来。扩大了法庭的司法权限。

\subsubsection{《大宪章》的签订与议会的召开}
1215年“失地王”约翰被迫签署《大宪章》

主要内容:限制王权,保障贵族,骑士和市民的权益;

《大宪章》以法律约束王权,王位与法律,对维护封建秩序有积极作用;保障市民的商业权利,有利于城市经济的发展;促进了英国议会;

开端:
\begin{enumerate}
    \item 1265年召开贵族,骑士和市民代表参加的全国性大会;
    \item 1295年议会被称为“模范议会”,标志着英国等级君主制的形成(由贵族议会到等级议会);
\end{enumerate}

\subsubsection{英国议会政治的发展}
最初为监督《自由大宪章》和保障贵族,骑士和市民权益的政治机构,后权限逐渐扩大。1297年获得批准征税的权力。14世纪初获得颁布法律和审理政治案件的权力。1343年,议会形成两院制,其中上院由贵族代表组成,下院由骑士和贵族代表组成。

\subsection{中世纪的西欧文化}

\subsubsection{中世纪的教育}
中世纪开始兴办大学,大多由神院发展而成。“大学”意为“总和”“联合”,即由学生和教师组成的行会。“院系”意为“才能”。中世纪大学不隶属教会,不依附地方,由学费经营。现在的学位制度源自中世纪欧洲。1500年全欧80所大学。

文学主要以拉丁文学和方言文学为主。特点是浓厚的宗教色彩。早期以拉丁语来创作。在11~13世纪,方言文学兴起,以英雄史诗,骑士抒情诗和骑士传奇与寓言为主。如法国的《罗兰之歌》,德意志的《尼贝龙根之歌》,英国的《亚瑟王传奇》。

\subsubsection{中世纪的建筑}
主要建筑风格为罗马式和哥特式。11世纪时,西欧经济发展,封建经济稳固,作为社会精神支柱的教会势力发展很快,不断兴建教堂和修道院。为了追求壮观的效果,这些建筑普遍采用类似古罗马建筑中拱顶与梁柱的结合,这中新的建筑样式被称为罗马式建筑。罗马式建筑多使用石头屋顶和圆拱,并用复杂的骨架结构来建筑拱顶。教堂平面设计的十字架形,是罗马式建筑的主要特点。罗马式教堂的外形像封建领主的城堡,窗户很小,而且距地面较高。

12到15世纪,欧洲盛行哥特式建筑风格。意大利学者认为,这一时期的建筑缺乏艺术趣味,所以用“蛮族”哥特人一词,称之为哥特式、哥特式建筑由罗马式建筑发展而来,但已不是城堡的样式。它的造型风格以高,直,尖和强烈的向上感为特征。哥特式建筑结构轻盈,纤细,大量使用小尖塔,并装有五颜六色的彩色玻璃。罗马式出现在早期9-12世纪,其中穹顶,后壁,窄窗,圆柱为主要特色。光线黯淡,给人以神秘感,渗透以罗马理性,实用的生活态度。哥特式出现在12-15世纪,高,直,尖,体现对天国的向往。薄墙壁,大窗户,高耸的尖塔,修长的石柱,门窗往往饰有彩色玻璃,内部光线明亮,以巴黎圣母院为典型代表。

