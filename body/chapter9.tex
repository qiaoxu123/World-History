\section{工业革命}

\subsection{英国工业革命的发生}
背景:18世纪末19世纪初在欧洲大陆上发生的政治革命和经济革命,其中政治革命主要对于欧洲大陆而言,而经济革命则主要对于英国本土而言。

\subsubsection{英国工业革命的前提}
政治革命:光荣革命后英国本土内资产阶级政权稳固,颁布了一系列法令促进了资本主义工商业的发展;

农业变革:改造农村举措推动了农业资本主义的发展,从而农产品的生产和加工得到了一定保障,客观上为大工业革命提供了物质保证。同时,生产力的提高,使得大量农村人口不断流入城市,为工业革命提供了劳动力;

海上霸权和殖民掠夺活动:为英国本土提供了可靠的资金,同时为产品提供了充足的市场以及原材料;

工场手工业的发展:工场生产中逐步形成的精细分工为机器的使用创造了条件。

\subsubsection{进程}
首先从棉纺织业开始,原因部分在于工业革命前期该行业不受行会束缚,自由竞争比较充分,从而促进了机器生产的出现。

约翰·凯伊发明“飞梭”;2.18世纪60年代哈格里夫斯发明“珍妮纺纱机”,引发了棉纺织生产领域一系列的发明;3.阿克莱特发明水力纺纱厂(近代机器大工业诞生的标志);4.克伦普顿发明了走锭纺纱机(螺机)1779年;5.卡特亨特发明水力织布机(1785年),将效率提高40倍;6.美国人惠特尼发明了轧棉机(1793年)。至此棉纺织页的流程基本实行了自动或半自动机器生产。

\subsection{英国工业革命的成就}
面临的挑战:动力与原材料(即后来的蒸汽机与冶铁技术)。工业革命前蒸汽机便已诞生(纽可门蒸汽机,1711年)。瓦特改良后发明“万能蒸汽机”(1782年)效率高,安全可靠。而原材料包含两部分,一部分是驱动蒸汽机的煤,另一部分则是制造蒸汽机的铁矿石。于是促进了开采技术和冶铁技术的进步。同时,铁矿石也广泛用于交通建筑中,如铁桥,铁轨等。
瓦特蒸汽机的出现推动了交通运输工具的进步。交通运输业是工业革命的集大成者,包括蒸汽机车,火车,轮船等的发明。史蒂芬孙发明蒸汽机车,美国人富尔顿发明了汽船(1807年)。

影响:英国逐渐成为世纪工厂,资本遍布世界,同时导致英国经济布局变动,改变了经济地理版图。在这之前,英国地理分布不均匀。在工业革命后主要城市主要分布在煤炭丰富地区,以及铁矿石丰富地区等。同时,相伴而来的还有城市化的进程,改变了农业社会的结构,开始出现新的阶级——工人阶级。工人阶级不断发展壮大,深刻改变英国社会。

1840年前后,英国大机器生产已经成为工业生产的主要方式,工业革命完成。工业革命创造的巨大生产力,使社会面貌发生了翻天覆地的变化。工业革命之后,资本主义最终战胜了封建主义,率先完成了工业革命的西方资本主义国家逐步确立起对世界的统治,世界形成了西方先进,东方落后的局面。

\subsection{工业革命在欧洲的扩展}
法国,德意志(由于长期分裂,起步较晚),也从棉纺织业开始,但与英国相比,缺乏稳定的市场,壁垒林立。后德意志关税同盟成立(1834年),1835年德意志铁路通车,促进了相关工业的发展;

俄国:发展极为缓慢,外国资本与技术起到了重要作用,发展不平衡,集中于欧洲部分。

\subsection{美国的早期工业化}
首先有棉纺织开始,由于英国严禁机器出口,塞缪尔·斯莱特(1768~1835),作为一名高级技工,化妆来到纽约,将棉纺织的先进技术带到了美国。后来惠特尼发明了轧棉机。

美国纺织业的发展,促进了南方植免业的发展,并在一定程度上加剧了奴隶贸易。

富尔顿于1807年发明汽船后,交通运输业充当了先导行业。并且美国也较早地实行了专利制度,促进了工业发明的产生。

惠特尼不仅发明了轧棉机,而且还是“美国规模生产之父”,创建了标准化额生产方式,改变了传统的工业生产,传遍美国。

美国的工业革命与西进运动同时发展,美国农业逐步走向机械化和高效率。

\subsection{19世纪初欧洲的社会思潮}
背景:法国大革命促进欧洲的思潮的涌现。

\subsubsection{自由主义(影响最大)}
 以自有作为主要政治价值。追求发展,相信人类善良本性,拥护个人自治权,主张放宽及免除专制政权对个人的控制。

来源:1776年,亚当·斯密的《国富论》(《国民财富的性质和原因的研究》),创立了第一个系统的政治经济学体系。

\subsubsection{民族主义(传播广泛)}
主张各民族都有权利保持和发扬自己的语言,历史和文化传统,并在政治上建立统一的民族主权国家。

中世纪时期还没有民族国家的概念,法国大革命时期积极传播的国家观念,包括国歌,国旗等,拿破仑也采取措施促进了民族主义的发展。

\subsubsection{保守主义}
反对激进的变革和进步,注重秩序和平

政治思想家爱伯蒙·伯克《法国革命论》(又译《关于法国革命的感想(1790)》),或曾被波旁王朝和保守势力所应用。

\subsubsection{空想社会主义}
(代表作:托马斯著《乌托邦》)

代表人物多为法国人,是由于法国历史革命未完成的背景。欧文作为工场主也曾实地实践过,以失败告终。原因包括不能说明资本主义的生产规律,以及没有科学的社会主义理论指导等。但为社会主义学说的建立以及开展社会主义的实践提供了重要的参考。

\subsubsection{浪漫主义(非政治性,偏文学与艺术)}

主要是一种文学和艺术的理论,涉及文学,哲学和艺术等领域。特点:1.偏重抒发强烈的个人感情;2.崇尚自然,描写自然风光;3.乐于描绘异域情调;4.擅长历史题材,美化历史;5.重视民间文学。代表人物:华兹华斯,歌德,雪莱,席勒等。

\subsection{马克思主义的诞生}
背景:随着工业革命的推进,新兴的工人阶级的利益不断被剥削。工业化早期,工人的状况悲惨,劳资矛盾尖锐;工人阶级开始了防抗,如英国的破坏机器运动,史称“卢德运动”,而欧陆工人采用武装起义,这种激进的斗争方式。工人们开始有秩序地组成工会组织,有组织地进行经济政治斗争。

三大工人运动;1.法国里昂工人武装起义(1831~1834年);2.德意志西里西亚织工起义(1844年);3.英国宪章运动(1838~1848)

马克思主义的三大来源

\begin{itemize}
    \item 德国的古典哲学(黑格尔的辩证法和费尔巴哈的唯物主义);
    \item 英国古典政治经济学(亚当·斯密和大卫·李嘉图);
    \item 空想社会主义学说(圣西门,傅立叶和欧文);
\end{itemize}

科学社会主义的创立 
1.1844年《神圣家族》;2.1845~1846年《德意志意识形态》;3.1847年《雇佣劳动与资本》;4.“正义者同盟”——共产主义者同盟。

马克思和恩格斯在长达40年的合作中,不断对社会主义进行探索,总结经验。为共产主义者联盟起草《共产主义宣言》。1848年发表。《共产主义宣言》分析了阶级斗争在阶级社会历史发展中的作用,揭示了资本主义必然要被社会主义代替的客观规律,号召全世界无产主义者联合起来,为获得自己的解放而斗争。《共产党宣言》的发表标志着马克思主义的诞生。从此,在科学理论的直到下,国际工人运动进入了一个新的历史时期。

1836年~1848年间,英国工人掀起了一场规模宏大,持续时间长久的运动。这次运动有一个政治纲领《人民宪章》,因此得名为宪章运动。宪章运动中,工人们要求获得普选权,以便有机会参与国家的管理。宪章运动得到了广发的支持,是世界上第一次群众性的,政治性的无产阶级革命运动。但是,由于缺乏科学理论的指导,工人运动的进一步发展受到了影响。(工人们把自己的要求以《人民宪章》的形式发表,以向议会递交请愿书的方式进行。持续了10多年)




