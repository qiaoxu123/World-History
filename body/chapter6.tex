\section{中世纪的东欧与东方世界}
\subsection{拜占庭帝国}
东罗马帝国,因首都君士坦丁堡原名为拜占庭,故称为拜占庭帝国。

\subsubsection{查士丁尼时代}
6世纪,东罗马帝国为恢复过去的罗马帝国版图,对外发动多次战争,地中海重新成为罗马帝国内海。但长期的战争造成国库空虚,财尽民穷。后来君士坦丁堡爆发瘟疫,从而功亏一篑。

\subsubsection{罗马法}
《民法大全》包括《查士丁尼法典》,《法学汇编》,《法理概要》和《查士丁尼新律》。是欧洲历史上第一部系统完整的法典。罗马法体系最终完成。成为欧洲各国的法律范本。法典既是对古代罗马法制传统的总结,也是对自罗马到拜占庭时期政治经济发展演变的反映。

罗马法的意义:罗马法保护私有财产,提倡法律面前公民人人平等。这有利于调整社会和经济生活中的纠纷,化解社会矛盾。特别是在罗马帝国时期,罗马法律制度渗透到国家的各个角落,稳固帝国统治。但是,罗马法也保护奴隶制度,维护奴隶主对奴隶的剥削和压迫。

罗马法是欧洲历史上第一部比较系统完备的法律体系,影响广泛而深远。罗马法对近代欧美国家的立法和司法产生了重要影响。当代很多法律制度中的原则和做法,都可在罗马法中找到源头。

近代时期,资产阶级根据罗马法中的思想,制定出保障自己权益的法律。他们还利用和发展了罗马法中的思想和制度,作为反对封建制度,推进资本主义发展的有力武器。

\subsubsection{拜占庭帝国的危机与存亡}
7世纪阿拉伯人围攻君士坦丁堡,拜占庭人用其秘密武器希腊火成功击退。第四次十字军东征(1204年)占领君士坦丁堡,征服并统治拜占庭长达半个多世纪。

拜占庭帝国约公元1400年灭亡,存留近1000年。其贡献主要在于保留了西方文明的独立性,同时其存在保留了大量古希腊,罗马的艺术品及文学品。同时传播了基督教。


\subsection{俄国}
斯拉夫人向东扩散的一支。

基辅罗斯公国:1.诺曼人(Normans);2.罗斯人(Rus);3.俄罗斯之源(Russia)

瓦里亚格人中的留里克兄弟(管理天赋)
诺夫哥罗德 862年建立了古罗斯国
留里克兄弟应邀来到诺夫哥罗德

拜占庭对罗斯人形象最大的是宗教

弗拉基米尔受洗(988年) 俄罗斯人皈依基督教。接受东正教为俄罗斯国教,从此俄罗斯欧洲化开始了。并且接受了西里尔字母(俄罗斯文学发展起来)

蒙古人入侵“金帐汗国(1240~1480)” 莫斯科公国的兴起

伊凡四世时,俄罗斯鼎盛。

\subsection{阿拉伯帝国}
6世纪末7世纪初,阿拉伯半岛还未形成统一的国家。为争夺水源和牧场,部落间相互仇杀。内部矛盾丛生,商道不畅,引起商业衰落,生产停滞和外部势力入侵。

阿拉伯帝国起源于麦加中心的克尔伯古庙(天房),供奉着一块陨石,阿拉伯来朝拜之人络绎不绝,并在此定期举办集市,由此麦加的地位日益上升,并由此与西欧等国家进行文化交流。在此之前,阿拉伯人信奉多神教,相信万物皆有生灵。但从西欧传来的一神教观念逐渐改变。穆哈默德趁此创立了与当地相适应的宗教——伊斯兰教。

创始人穆哈默德(570~632),伊斯兰一词原意是“皈服”,伊斯兰教徒称为“穆斯林”,意为“皈服者”,即信仰安拉,服从先知的人。

伊斯兰教主要经典——《古兰经》(“古兰”即诵读之人),不仅阐述了宗教教义,而且包括了早期阿拉伯经济,政治,文化等内容。

伊斯兰教义包括两部分,分别是宗教信仰和宗教义务。宗教信仰“大信”,分别是信安拉,信天使,信使者,信经典,信后世,信前定。其中最主要的是信安拉,认为除安拉之外再无别的神灵。伊斯兰教宗教义务“五功”,分别是“念,拜,斋,课,朝”功。斋功指伊斯兰历9月一天,禁止饮食;课功指宗教税,约为财产的四十分之一,用过济贫税;朝功指去麦加圣地朝拜,对团结全世界的穆斯林有重要作用。

穆罕默德建立了麦地那,后与麦加贵族妥协,麦地那为国家首都,麦加为宗教中心。穆哈默德去世后,为“四大哈里发”时期,倭马亚王朝(白衣大食),阿拔斯王朝(黑衣大食)时经济发展达到全盛,此时阿拉伯成为地跨欧亚非三大陆的新王朝,重视农业与贸易,农耕面积占全世界农耕面积的一半以上。阿拉伯对促进东西方文化交流起到了重要作用,并对文艺复兴产生过巨大推动作用。

\subsection{蒙古帝国}
蒙古帝国的首领“成吉思汗”(1206-1227),原名叫铁木真,成吉思汗意为“征服四方”。其推行“千户制”,是一种军事化政治组织。“上马则备战争,下马则屯聚牧养”。

1218~1260,蒙古西征,建立蒙古四大汗国。

对西方文明起到很大的破坏作用,但客观上也向欧洲传入了火炮等技术,促进了欧洲发展。

\subsection{印度}
印度是世界宗教发祥地之一,宗教文化渗透到生活的方方面面。

\subsubsection{佛教的产生}

创始人是乔达摩·悉达多(前566~486)“释迦摩尼” 觉悟者的意思。

佛教的主要教义是四谛“苦谛,集谛,灭谛和道谛”。还有八“正道”“正见,正思,正语,正业,正命,正精进,正念和正定”。佛教在印度的推广中起重要作用的是印度孔雀王朝的君主阿育王(前273~236),公元前3世纪,阿育王统一印度,最终弘扬了教义。

佛教宣扬“众生平等”,反对婆罗门的特权地位。它认为时间万物发展都有因果缘由。人的生老病死都是苦,人必须消灭欲望,刻苦修行。很多国王利用它“忍耐服从”的说教,大力扶持佛教。公元前3世纪,阿育王那个在位时,佛教有了很大发展,并向外传播。

佛教主要向两个方向传播:向北,经中亚地区传到中国大部分地区,以后又从中国传到朝鲜,日本和越南等国;向南,传入斯里兰卡,泰国,缅甸等国和我国境内傣族地区。

\subsubsection{印度教的产生}
印度的三大宗教分别是婆罗门教,佛教和印度教。婆罗门教宣扬种族制度,而佛教宣扬灭欲,印度教则融合了两种宗教的特点。

后来伊斯兰教入侵(加兹尼王朝的马哈茂德(998~1030)),称为“偶像破坏者”。后来伊斯兰教仅次于印度教,后定位国教,长达几个世纪。

\subsection{日本}
日本列岛地形上属于湖形列岛。各岛很早就有人类居住。1世纪前后,开始出现奴隶制国家。后来,本州中部兴起奴隶制国家大和。大和不断进行扩张征服,5世纪时统一了日本。

\subsubsection{大化革新}
六七世纪时,日本的社会矛盾十分尖锐,大贵族奴隶主势力强大,政局混乱,改革势在必行。7世纪中期,改革派发动宫廷政变成功,新上台执政的孝德天皇颁布改新诏书。

中世纪的日本“倭”大和民族(天照大神),实行部民制。孝德天皇时实行改革,因为孝德天皇的年号是大化,因此称这次改革为“大化改新”。

大和的世袭大贵族占有大量土地,奴隶以及地位近似奴隶的部民,权势极大。6世纪末7世纪初,部民反抗不断;地方贵族反抗中央贵族,中央贵族之间争夺权势;强大的中央贵族权倾朝野。中央唐朝的统一和兴盛,朝鲜半岛上新罗国家的崛起,强烈刺激了大和统治者。天皇和一些留学中国的士大夫决定参照隋唐制度,实行改革。

大化革新开始于646年,措施包括:废除王室和一切贵族的私有领地和部民,土地收归国有,部民成为国家的公民,即“公国公民”;实行班田收授制,国家对公民班给口分田,六年一次,死后归还。受田者要提租庸调;改革官制,建立中央集权的国家机构,天皇制封建国家,各级官吏由国家任免,废除世袭制。以才选官。

这次自上而下的政治改革,逐步将日本由奴隶社会变成封建社会。由部民制逐渐成为中央集权国家。班田制成为国家主要生产方式。大化改新后,大和正式改名为日本国,意为“日出之处的国家”。

\subsubsection{武士阶层}
大化改新后一二百年,日本地方豪强崛起,他们占有大量土地,建起庄园。为保护自己的庄园和加强统治,豪强把自己的家族和仆人武装并蓄养起来,组成一种以血缘关系和主仆关系为纽带的军事集团,特殊的武士阶层逐渐形成。

武士必须对主任忠贞不二,充当主人的私人武装。庄园主有义务供养,保护武士。武士的主要职责是保卫庄园,平息地方贵族叛乱,镇压人民起义,在日本封建社会历史中发挥了极为重要的作用。
武士道是日本军事封建专制主义的产物。

\subsubsection{日本封建制度(庄园制度)}
由于在原本田地之外,王室鼓励农民开垦私田,允诺私田归个人所有。于是大块荒地被开采出来,但同时私田制下贵族和寺院不断利用职权取得土地,逐步形成庄园制度。加之贵族不断要求变更政策,逐渐庄园成为不受政府支配的独立区域。12世纪时班田制彻底瓦解。

庄园生活基本上可以自给自足,但庄民需要承担各种义务。庄园武装庄民,形成“武士”。征夷大将军源赖朝(1192~1199)建立幕府(武士贵族专政时期),成为事实上的中央政府。在这时期形成武士道精神,“忠勇”为武士道的核心。

战国时代后日本统一,德川幕府(1603~1868)专制制度重新确立。经历了几个阶段:日本的重新统一;织田信长(1534~1582);丰臣秀吉(1537~1598);德川家康(1542~1616)。

德川幕府加强中央集权统治的政策:1.划定将军直辖领地;2.公布“武家诸法度”;3.继承和发展了身份等级制度“四民(士农工商)”;4.禁教锁国;

