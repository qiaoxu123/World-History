\section{大民族国家的巩固}

\subsection{英国的自由主义改革}
背景:随着工业革命的开展,工业资本主义蓬勃发展,并且不断产生新兴城市。这就迫切要求在政治上取得一席之地,以进一步改革国家政策,促进工业资本主义的发展。

英国议会改革(原先选举制度存在种种不合理,包括席位等。迫于国内的政治压力,政策进行了改革)

\subsubsection{第一次议会改革(1832年)}
内容;1.调整选区和重新分配各选区议员名额;2.56个“衰败选取”被取消;3.31个选区减少1个议员席位;4.各大新兴工业城市得到65个新的席位;5.选民财产资格限制。
而社会的发展则提出更高的要求,广大公民迫切获得选举权,通过政策来改变自己的生存的现状,如宪章运动。

\subsubsection{第二次议会改革(1867年)}
内容:1.取消46个衰败选区在下院的席位,将其转给工业城市;2.降低选民财产资格;3.选民从136万人(1866)扩大到246万人(1868)。

\subsubsection{英国文官制改革}
之前的政府机构中,存在各种混乱,官员舞弊等现象,严重影响了政府机构的效率以及政策制定等问题。

威廉·格拉斯顿提出改革,即第三次议会改革。

内容:1.1853年《关于建立常任文官制度的报告》;2.50~70年代,文官制度改革的基本依据;3.1870年关于文官制度改革的命令,规定以公开竞争考试来录用文官,建立了公开竞争考试的原则。
通过自有竞争方式竞选文官,从而使国家政府运行更有效率,更加民主,从而促进了工业资本主义的发展,并对日本等其他国家产生深远影响。


\subsection{法国共和制的确立}
拿破仑帝国经过了波旁王朝复辟,七月王朝之后,拿破仑侄子路易·拿破仑上台,企图恢复帝制,用武力解散议会,后登基为皇帝,始称拿破仑三世。

拿破仑三世的统治森严,用警察,暗探和军队进行高压统治。但为长期动乱的法国社会提供了难得的稳定,促进了法国工业革命的发展,并于1867年在巴黎举办了世博会。

1870年,法国同普鲁士发生战争,法军战败,普鲁士士兵兵临巴黎城下,资产阶级政府对外屈膝投降,对内准备镇压人民。1871年3月,政府军队同巴黎市民武装——国民自卫军发生冲突,导致巴黎工人起义爆发。起义工人很快占领了全城,赶走了资产阶级政府。不久,人民选举产生了自己的政权——巴黎公社。巴黎公社成立后,建立起自己的军队和行政,司法机构,还制订了许多保护工人利益的政策。资产阶级政府不甘心失败,对巴黎公社发动了进攻。5月21日~28日,公社战士同攻入巴黎城内的敌人展开了激烈的巷战。最终英勇牺牲。巴黎公社是无产阶级建立政权的一次伟大尝试。

不论是普法战争还是后来的巴黎公社,都提出同一个问题,即法国实行何种制度的问题,即君主制还是共和制,后法国确立了共和制。

法国共和制的确立

总统选举方式:“共和国总统”应由参议,众议两院选举产生;1875年宪法,即法兰西第三共和国宪法。

共和党上台后,原先议会中的王党派,保守派被清洗,共和制最终确立。

\subsection{德国的统一}
德意志在其它欧洲国家迅速进行工业化发展的时期,却一直处于割据状态,包括普鲁士和奥地利。在法国大革命时期,拿破仑的征服促进了其资本主义政治的改革,并且,在被征服地区,民族意识空前崛起。

维也纳体系后,德意志成为了松散的联盟。在工业革命时期,从英国引进机器促进了本国工业的发展。同时于1834年成立了德意志的关税同盟,奠定了德意志统一的经济基础。

在德意志的联邦中,普鲁士经济实力发展最快,在统一的德意志联邦中扮演着重要角色。奥托·冯·俾斯麦担任普鲁士宰相后,开始实行“铁血政策”,希望通过武力来实现德意志联邦的统一。

在国内积极加强军备建设,在外交上运用各种手段为统一德意志创造条件。1864年普鲁士与奥地利共同对丹麦宣战,丹麦战败。奥地利取得了赫尔施坦,普鲁士取得了石勒苏益格。史称第一次王朝战争。在战争中窥探了奥军的实力,同时巩固了俾斯麦的地位。

普奥战争(1866)最后奥地利战败,签订了布拉格合约(奥地利退出德意志邦联,普鲁士建立了北德意志联邦。剩余各邦在法国的阻隔下不能统一,于是俾斯麦以爱姆斯电报为由,对法宣战。法国打败投降,普鲁士割取法国的阿尔萨斯和洛林,并获得赔款50亿法郎。1781年1月18日,在法国凡尔赛宫镜厅,统一的德意志帝国诞生。

影响:统一的德意志帝国为资本主义发展扫清了障碍,改变了欧洲的地缘政治格局,为国内,国外创造了良好氛围。由于王朝战争自上而下进行改革,德意志联邦内的容克贵族和军国势力被保留下来,统一后仍具有历史保留性。

\subsection{意大利的统一}
意大利民族复兴运动(政治分裂,外族压迫——意大利迈向现代文明的两大障碍)

奥地利控制伦巴蒂和威尼斯;2.西班牙波旁王朝控制那不勒斯;3.半岛中部“教皇国”;4.西北部的撒丁王朝。

民族复兴运动始于秘密组织(烧炭党:由各个阶层组成),后起义被镇压,影响减弱。(原因:意大利资产阶级薄弱,加上外国干涉)。

北部意大利王国撒丁独立,首相加富尔进行改革,成为意大利最先进的国家,成为日后统一意大利的中心。与法国联盟对抗奥地利,战胜后成立意大利联邦,意大利实现局部统一。

加里波第“红衫军”支持南意大利的起义,意大利南部基本统一。后在其他国家战争影响下,采取措施收复了威尼斯和罗马,形成统一。

\subsection{俄国农奴制的改革}
俄国农奴制严重阻碍了社会发展,农业生产低下,生产方式落后,资本主义缺乏劳动力,限制了国内市场的扩大。俄国面临农奴制危机。

克里米亚战争,俄国与土耳其宣战,后英法担心其势力,宣战俄国。俄国战败,签订《巴黎条约》(1853~1856),暴露了农奴制下俄国的腐朽,使社会矛盾越来越尖锐。

1861年废黜农奴制,后又实行资产阶级改革,建立了资产阶级机构。进行司法和财政改革。改革后为地主和资产阶级的联合统治,从封建到资产阶级的转变是不彻底的。

\subsection{美国内战}
背景:美国独立后,不断进行领土扩张,经济迅速发展,但南北方存在着制度上的根本差异。北方实行资本主义工商业,南方则实行的是种植园奴隶制,并且各有发展。工业革命时期对棉花的需求致使南方种植园迅速发展,奴隶制的迅速发展也加深了南北双方的矛盾。同时北方资产阶级为了发展生产,需要大批劳动力,但南部奴隶制的存在,占用了大量的劳动力。

密苏里州妥协案(对于新的州的制度选择问题:蓄奴制或资本制)
由于州在国会占有席位,因而演变为对国会控制权的争夺。
双方矛盾还表现在关税问题上,南方种植业希望扩大对外贸易,降低关税;而北方则希望能稳定国内市场,提高关税。
同时奴隶的生存状况不断引起北方,反奴隶制人士的担忧,如“地下铁道”路线,斯托夫人《汤姆叔叔的小屋》等,更有起义者讨伐奴隶制。《堪萨斯·内布拉斯加法案》(1854),南北双方不断爆发冲突。

1860年11月,林肯当选总统,代表的美国共和党1860年总统竞选的主题是:联邦必须而且将会得到保留。威胁奴隶主们的利益。成为南方奴隶主发动叛乱的接口,南方一些州联合起来,宣布组成一个独立国家,号称“南部联盟”,之后南北战争爆发,分为两个阶段。

初期南方联邦掌握主动权,并且组建一支训练有素的军队,而北方联邦在林肯总统的领导下,没有发挥出北方联邦道义,工业上的优势。主要是由于联邦政府的犹豫与动摇,对于国家统一的目的而不是废除奴隶制,林肯只是希望南方奴隶制能不再扩张,恢复联邦统一。同时担忧少数实行蓄奴制的中立联邦因为奴隶制的废存而加入南部联邦。战局北方劣势。林肯同情黑人奴隶,反对奴隶制度,但担心如果处理不当,会激化南北矛盾,造成国家分裂。所以,他不主张立即废除奴隶制,而是逐步限制奴隶制的发展。

迫于国内对于劣势局势的不满和奴隶制的怨恨,林肯政府采取措施,颁布《宅地法》与《解放黑奴宣言》。规定1863年元旦起,废除叛乱各州的奴隶制,并允许奴隶作为自由人参加北方军队。广大黑人欢庆解放,踊跃报名参军,北方军队因此获得雄厚的兵源。战争性质骤然改变,由维护国家统一变为废除奴隶制之战。北方军队转为主动并最终战胜南方联邦。

\subsection{美国南部重建}
南北战争胜利后,林肯制订了一系列南方重建法案,但不幸遇刺。之后上台总统基本上实行了林肯的南方重建措施,但对于叛乱者过于宽松,之后甚至恢复其社会地位与财产,从而南部奴隶主重新得势,秘密地对黑人进行迫害。经过之后的一系列法案,给予黑人以选举权和公民权。

美国内战和南方重建最后以资产阶级与奴隶主的妥协告终,但全国基本实现了资产阶级统治,为发展扫清了障碍,美国步入迅速发展时期。南北战争是美国历史上第二次资产阶级革命。经过这场战争,美国废除了奴隶制度,扫清了资本主义发展的又一大障碍,为以后经济的迅速发展创造了条件。

\subsection{日本明治维新}
19世纪中期的日本,仍是闭关锁国,落后的封建国家。天皇大权旁落,权力掌握在幕府将军手中。将军的政厅被称为幕府。19世纪中期以前,幕府将军掌握大权的情况延续几百年。

19世纪后半期欧美各国开始将目光投降亚洲,日本与其它国家一样面临民族危机。同时日本正处于幕府统治时期,封建社会末期的种种矛盾不断发生,下层群众生活艰难,不断爆发起义。同时资本主义势力开始联合,要求改革,反对幕府统治,

在日本闭关锁国时期,只有荷兰与之贸易,并传入资本主义思想和科学技术,被称为“兰学”。一部分精通兰学的有识之士成为日后明治维新的领导力量。

1853年美国海军准将柏利率舰队抵达日本,迫使日本开放港口,签订条约,结束了日本闭关锁国的时代。外国势力的渗透,激化了日本的国内矛盾,幕府的统治岌岌可危。一部分中下级武士,逐渐放弃了排斥西方的做法,开始接受西方的先进技术和思想,主张以武力推翻幕府的统治,并取得成功。国内开始产生“倒幕派”,并且通过途径控制了皇宫,废除了幕府。后德川庆喜企图用武力镇压倒幕派,后失败,各藩藩主开始忠于倒幕派,幕府统治被彻底推翻。开始实行明治维新,以天皇名义进行自上而下的改革。

1867年,以一部分中下级武士为首的改革派,从年幼的天皇手中得到了讨伐幕府的密诏。幕府将军德川庆喜迫于形势,假意把政权还给天皇,却不肯交出兵权和领地。第二年,讨伐幕府的武装在京都附近的战斗中,战胜幕府军队,取得了决定性胜利,幕府统治被推翻。1869年,明治天皇政府从京都迁到东京。

推翻幕府后,明治天皇政府实行了一系列资产阶级性质的改革,主要内容有:政治方面:“废藩置县”,加强中央集权;经济方面:允许土地买卖,引进西方技术,鼓励发展近代工业;社会生活方面:提倡文明开化,即向欧美学习,努力发展教育。这些改革是在明治年间进行的,因此被称为“明治维新”。

在改革中,明治政府宣布士农工商平等。废除贵族称号,改称华族;武士改称士族;农民,商人,手工业者统称为平民。允许平民有选择职业的自由;允许华族,士族和平民之间通婚。原来的封建等级制不复存在。

为迅速发展工商业,明治政府兴建工厂,修筑铁路,举办邮政,电讯事业,扶植,保护私人企业的发展。为“求知识于世界”,日本还请来大批欧美专家和技师。

明治政府重视发展教育,规定了统一的学制;仿照欧美,设立了新式学校,普及初等教育。政府提倡学习欧美的资本主义文明,盖洋楼,吃西餐,穿西服,改变传统的日本发型。

明治政府开始实行征兵制,建立一支崇尚“武士道”精神,效忠天皇的军队。这支新建立的军队,很快成为日本对外侵略扩张的工具。

明治维新使日本从一个闭关锁国的封建国家,逐步转变为资本主义国家,摆脱了沦为半殖民国家的命运,是日本历史的重大转折点。但日本强大起来以后,很快就走上了对外侵略扩张的军国主义道路。

\subsubsection{明治政府的改革措施}
1.废除幕藩封建体制,建立以天皇为中心的中央集权制。废藩设县,废除封建等级身份制,取消武士特权。

2.改革土地和地税制度,建立新的土地制度。承认土地私有。

3.“殖产兴业”,“文明开化”,“富国强兵”三大政策。

4.明治政府大力发展近代资本主义工商业;大力发展近代教育,培养资本主义建设人才;设立文部省,系统引进西方教育体系;建立征兵制和常备军(灌输军国主义思想和武士道精神,“一切忠于天皇”)

但明治政府仍为专制主义统治,后人民开展自有民主运动,明治政府于1889年颁布了帝国主义宪法,确立了君主立宪政体,成为亚洲唯一独立进行资产阶级改革国家,但保留了封建残留。

